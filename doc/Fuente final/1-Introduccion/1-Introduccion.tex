\chapter{Introducción}

   \section{Contexto del problema}
   La finalidad del software que se va a desarrollar es que sea capaz de ejecutar una serie de órdenes que introduzca el usuario. Estas ordenes
   permitiran mover y articular la representación de un humanoide (stickman). Éste debe poseer ciertos conocimientos de programación para que le sea más 
   fácil la utilización adecuada de sentencias, bucles, interfaz, etc.\\

   La aplicación sólo reconoce y acepta sentencias desarrolladas para el lenguaje de programación \textbf{\textit{Sticky}}, el cual se abordará
   en posteriores secciones de este documento.\\


   \section{Arquitectura de la aplicación}
   La aplicación a desarrollar será posible ejecutarla sólo en la máquina local, por lo que no requerirá ningún tipo de conexión a Internet ni
   dependerá de otra aplicación, módulo o similar. De esta forma se hace sencillo su movilidad entre diversas máquinas y entre diversos sistemas
   operativos, pues al estar desarrollada en Java permite una máxima compatibilidad, aumentando de esta forma el número potencial de usuarios que
   utilicen esta herramienta.\\

   \section{Descripción de la aplicación}
   La aplicación, de escritorio como se ha mencionado anteriormente, consiste en una interfaz gráfica que tiene dos partes claramente diferenciadas: 
   \begin{enumerate}
      \item La parte donde se dibujará Stickman y éste realizará todos sus movimientos. 
      \item La parte donde se introducirán las sentencias o instrucciones; en definitiva, el uso de la aplicación.\\
   \end{enumerate}

   Para introducir las instrucciones se proveerán dos métodos: 
   \begin{enumerate}
      \item Entrada estándar desde la propia aplicación, lo que facilita al usuario la prueba y testeo de acciones simples. 
      \item Entrada desde fichero, lo que permite al usuario la prueba y testeo de un conjunto de instrucciones que conforman una acción más compleja.\\
   \end{enumerate}















