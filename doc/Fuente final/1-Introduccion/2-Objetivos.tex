\chapter{Objetivos}
Bla, bla, bla...


   \section{Generales}

   \section{3D -> Discutir} % ---------------------------------------------------------------------

   \section{De la gramática}
   A continuación se describen los objetivos de la gramática para el reconocimiento del lenguaje Sticky que se va a desarrollar: 
   \begin{itemize}
      \item Todas las instrucciones del lenguaje llevarán el carácter fín de instrucción (’;’). 
      \item Los operadores aritméticos reconocidos son: +, -, *, /, ++, –, %, ^, raiz(x). 
      \item El lenguaje será sensible a las mayúsculas y minúsculas.
      \item Tipado dinámico de datos: 
            \begin{itemize}
               \item Definición de variables: var nombreVariable. 
               \item Destrucción de variables: sup nombreVariable. 
               \item Tipos de variables soportadas: booleanos, enteros, reales y cadenas. 
            \end{itemize}
      \item Comentarios: monolínea (’//’) y multilínea (’/*’ y ’*/’). 
      \item Asignaciones: símbolo igual (’=’). 
      \item Operadores condicionales: ’<’, ’>’, ’<=’, ’>=’, ’==’ y ’!=’. 
      \item Operadores lógicos: ’Y’, ’O’, ’OX’, ’NO’, ’!’. 
      \item Los bloques de código irán encerrados entre llaves (’{’ y ’}’).
      \item Construcción de bucles: 
            \begin{itemize}
               \item Bucle for (para):
                     \begin{itemize}
                        \item para (inicio; fin; incremento) { sentencias } 
                        \item para (inicio; fin; incremento) sentencia 
                     \end{itemize}
               \item Bucle while (mientras):
                     \begin{itemize}
                        \item mientras (expresión o variable) { sentencias } 
                        \item mientras (expresión o variable) sentencia 
                     \end{itemize}
            \end{itemize}
      \item Estructuras condicionales:
            \begin{itemize}
               \item if (si):
                     \begin{itemize}
                        \item si (condición) 
                        \item si (condición) { sentencias } 
                     \end{itemize}
               \item if-else (si-sino):
                     \begin{itemize}
                        \item si (condición) sentencia sino sentencia 
                        \item si (condición) { sentencias } sino sentencia 
                        \item si (condición) sentencia sino { sentencias } 
                        \item si (condición) { sentencias } sino { sentencias } 
                     \end{itemize}
            \end{itemize}
      \item Switch:
%             \begin{itemize}
%                opcion (expresión o variable) { 
%                   caso expresión1: { sentencias } fincaso; 
%                   caso expresión2: { sentencias } fincaso; 
%                      ... 
%                   defecto: { sentencias } fincaso; 
%                } 
%             \end{itemize}
      \item Movimientos propios de \textit{StickMan}:
            \begin{itemize}
               \item girar CABEZA (angulo, angulo, tiempo);
               \item girar [BRAZO | PIERNA] [IZQ | DER] (angulo, angulo, tiempo);
               \item girar STICKMAN (angulo, tiempo);
               \item flexionar [BRAZO | PIERNA] [IZQ | DER] (angulo, tiempo);
               \item mover STICKMAN (x, y, z, tiempo);
            \end{itemize} 
      \item Funciones temporales:
            \begin{itemize}
               \item tiempo [ESTABLECE | AVANZA] (tiempo);
            \end{itemize}
   \end{itemize}






















