\chapter{Autoevaluación}
El objetivo fundamental de este apartado es el de valorar, criticar y reflexionar el comportamiento del grupo de trabajo con 
respecto a la estrategia seguida en el proceso de aprendizaje y desarrollo del sistema, además de facilitar al docente la compresión
del proceso anterior en relación a las dificultades acontecidas y los objetivos conseguidos. \\

En primer lugar, se hará evidencia de los \textbf{esfuerzos realizados} con respecto a los conocimientos. Para ello, se responderán a 
una serie de cuestiones que a continuación se enumeran.

   \section{¿Qué conocimientos había sobre el desarrollo de la aplicación?}
   Todos los miembros conocíamos el paradigma de programación orientado a objetos (POO), estudiado en la asignatura de Estructuras de Datos y
   de la Información en la Universidad de Córdoba, por lo que la adaptación a Java ha sido rápida. Con respecto al uso de ANTLR, conocíamos
   el uso de analizadores léxicos como Lex combinado con Yacc, el cuál es el encargado de realizar el análisis sintáctico. De la misma forma
   que en el caso anterior, el aprendizaje de las herramientas se realizó en la asignatura de Teoría de Autómatas y Lenguajes Formales.\\

   Con respecto a las herramientas de desarrollo 2D o 3D, conocíamos la existencia de Java2D, Java3D o VRML, pero ningún miembro del grupo
   tenía experiencia previa con dichos lenguajes. No obstante, hemos estudiado cada uno de los tres para valorar cual se adecuaba más a nuestras
   intenciones, eligiendo, como se comenta anteriormente en éste mismo documento, la API de Java 3D, lo cual conllevó a un estudio en más 
   profundidad del mismo para poder utilizarlo eficientemente y sin errores.


   \section{¿Cómo lo hemos aprendido?}
   Con respecto a ANTLR, además de realizarse un seguimiento en varias clases prácticas de la asignatura, se han consultado tanto los
   manuales proporcionados para su comprensión como información adicional de la web. Debido a que se realiza un uso extenso del mismo,
   el aprendizaje ha sido costoso, a pesar de haber manejado otras herramientas con un fin parecido.\\

   En el caso de Java y Java3D se han consultado referencias web, libros, la propia API, pequeños de desarrollos de prueba y ejemplos que
   nos proporcionó el profesor de la asignatura. En el caso de Java, el aprendizaje ha sido bastante rápido comparado con los demás elementos
   debido al uso anterior del lenguaje C++. En cambio, el uso Java3D ha requerido gran parte de tiempo debido al desconocimiento del mismo. 


   \section{¿Qué sabemos ahora, una vez acabada la aplicación?}
   Una vez contestadas las preguntas anteriores, podemos afirmar que se ha logrado la compresión de las herramientas anteriores (Java, 
   ANTLR, Java3D), lo cuál nos puede ayudar a progresar más rápidamente en futuras asignaturas o a poseer mayores conocimientos a la hora 
   de introducirnos en el duro y competitivo mercado laboral. Para ello, la organización temporal y la división de la carga de trabajo entre
   los diferentes miembros del grupo ha sido esencial, lo cual ha hecho que fomentemos un espíritu de cooperación entre nosotros, así como un
   estupendo ambiente de trabajo.\\[1.5cm]

   Por otra parte, se analizará el \textbf{trabajo en grupo} realizado para cumplir los objetivos del desarrollo de este sistema. Los integrantes
   del grupo establecimos reuniones semanales, las cuales contemplaban dos días de la semana, para dar a conocer a los demás integrantes
   la situación en la que se encontraba el trabajo individual realizado y la explicación del mismo si fuese necesario.\\

   Es necesario remarcar la distribución de trabajo realizada. En un principio, el hecho de no tener amplia experiencia con grupos de 
   trabajos con un elevado número de personas nos hizo pensar que la carga de trabajo individual no iba a ser proporcional, es decir,
   que unos miembros podrían realizar un mayor esfuerzo que otros. Gracias a \textit{SVN} y a \textit{GoogleCode}, esta repartición de tareas ha sido
   mucho más fácil y cómoda, ya que hemos podido ver los resultados de cada uno, tanto a la hora de codificar como de documentar, de forma 
   instantánea y sin solapar el trabajo de otro compañero. Además, SVN dispone de un log que indica las actualizaciones que ha realizado
   cada miembro del grupo, por lo que también hemos podido controlar el trabajo realizado por cada uno y comprobar que se ha hecho en la 
   fecha estimada en nuestras reuniones semanales.\\

   Finalmente, realizaremos una \textbf{evaluación de nuestro desempeño} en el trabajo considerando los siguientes indicadores cumplidos:
   \begin{itemize}
      \item La comunicación fue dinámica, activa y agradable, participando todos con sus aportaciones. Cuando alguien hablaba todos le escuchaban,
            asimilando la información.
      \item Las decisiones se tomaron por consenso con la participación de todos los miembros del grupo y como resultado de la aportación
            creativa de todos.
      \item Todos los miembros del grupo se autocontrolaron y se autodirigen.
      \item Como en cualquier trabajo en grupo, han aparecido numerosas discrepancias entra los miembros. Cuando éstas se presentaron, se
            analizaron objetivamente exponiendo argumentos razonables para optar por la mejor alternativa.
      \item En alguna ocasión el grupo se detuvo a analizar cómo estaban trabajando, detectando errores y acordando mejoras de acción.
      \item El ambiente de trabajo ha sido óptimo, fomentando la participación y el compromiso. 
      \item Durante el desarrollo de la aplicación, en ocasiones nos hemos visto obligados a rediseñar ciertas partes debido a malas 
            decisiones tomadas al inicio, aunque éstas fueron resultas adecuadamente.
   \end{itemize}

Debido a los puntos anteriores y siguiendo una escala de 0 a 5, la puntuación final dada al trabajo realizado es de 5. Cabe destacar que, 
además, se ha tenido en cuenta que hemos sido un grupo bastante numeroso y que se ha realizado un trabajo completo y original.