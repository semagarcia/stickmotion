\chapter{Conclusiones}
Una vez concluido el desarrollo de la aplicación, podemos afirmar que se ha conseguido un sistema en el que, a través de una 
serie de órdenes, se ejecutarán una serie de movimientos concretos y específicos sobre un objeto 3D denominado \textbf{\textit{Stickman}}.\\

A pesar de que el usuario de la aplicación debe poseer conocimientos básicos sobre programación, se ha desarrollado un entorno 
intuitivo y e fácil manejo, existiendo un apartado de ayuda que permita clarificar su uso.\\

Por lo tanto, al término de la fase de desarrollo de esta aplicación, las conclusiones más relevantes que se pueden extraer son las siguientes:
\begin{itemize}
   \item El sistema permite el movimiento de una figura 3D a través de una serie de comandos introducidos por el usuario.
   \item La aplicación sólo reconoce y acepta sentencias desarrolladas para el lenguaje de programación \textbf{\textit{Sticky}}.
   \item La ejecución de la aplicación se produce en un entorno local, por lo que no requiere conexión a Internet ni dependerá de otra 
         aplicación, módulo o similar.
   \item Al estar desarrollado en Java, se hace sencilla la portabilidad entre diferentes máquinas y diversos sistemas operativos, haciendo
         que el número potencial de usuarios que utilicen esta herramienta aumente por dicha compatibilidad.
   \item Se ha logrado una representación sencilla del cuerpo de una persona, la cuál posee sus características y articulaciones principales
         que podrán ser manipuladas.
   \item Se proporcionan una serie de funciones, específicas del lenguaje, que dotarán de cierto dinamismo al objeto 3D representado, tal y 
         como la flexión de brazos y piernas o la rotación del cuerpo.
   \item El lenguaje desarrollado sigue las pautas de la mayoría de los lenguajes de programación más populares, como por ejemplo C, lo que
         hace que el uso del sistema sea más cómodo para los programadores.
\end{itemize}

Por otro lado, la interfaz de la aplicación con la que el usuario interactúa presenta las siguientes conclusiones:
\begin{itemize}
   \item La barra de menú y herramientas proporcionan tanto las funcionalidades básicas de cualquier aplicación (como abrir un fichero o 
         deshacer la última acción) como aquellas específicas para el movimiento de nuestro \textbf{\textit{Stickman}} (como pudiera ser
         la inserción de esqueletos de código o un selector de nivel de depuración).
   \item El sistema proporciona un editor de texto que ayuda a la escritura y lectura de código que desarrolle el programador, ya que se realiza
         un coloreado de las palabras claves y se resaltan en cursiva las constantes.
   \item Los resultados de la lectura del código son mostrados en una ventana de animación, además de proporcionar información adicional al
         usuario en términos de interpretación.
\end{itemize}


