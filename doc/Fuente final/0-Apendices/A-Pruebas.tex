\documentclass[a4paper, 12pt]{book}

\usepackage{graphicx}
\usepackage[utf8x]{inputenc}
\usepackage[T1]{fontenc}

\begin{document}

\chapter{Pruebas y Códigos de ejemplo}
Para comprobar que la aplicación satisface todos los objetivos establecidos, y que los realiza de forma correcta, se somete a una serie de 
pruebas para determinar su fiabilidad y consistencia. En base a dichas pruebas, se realizan las pertinentes correcciones hasta obtener un sistema
completamente depurado y de calidad.

Dividiremos las pruebas en dos tipos:

\begin{itemize}
 \item Pruebas sobre el funcionamiento de la interfaz.
 \item Pruebas de ejecución.
\end{itemize}


\section{Pruebas sobre el funcionamiento de la interfaz}
Tal y como indica el título de este apartado, se realizarán pruebas sobre los distintos elementos de la interfaz, las cuáles se irán describiendo
en las diferentes secciones presentadas a continuación.

\subsection{Opciones de la barra de menú}
A continuación se detallarán las pruebas en relación a la barra de menús de la aplicación, así como los errores encontrados y las soluciones 
aportadas. Se contemplan los siguientes casos:

\begin{itemize}
  \item Visualización de los distintos apartados y subapartados (si procede).
  
\end{itemize}



\section{Pruebas de ejecución}


\end{document}


