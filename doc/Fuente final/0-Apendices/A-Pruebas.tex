\documentclass[a4paper, 12pt]{book}

\usepackage{graphicx}
\usepackage[utf8x]{inputenc}
\usepackage[T1]{fontenc}

\begin{document}

\chapter{Pruebas y Códigos de ejemplo}
Para comprobar que la aplicación satisface todos los objetivos establecidos, y que los realiza de forma correcta, se somete a una serie de 
pruebas para determinar su fiabilidad y consistencia. En base a dichas pruebas, se realizan las pertinentes correcciones hasta obtener un sistema
completamente depurado y de calidad.

Dividiremos las pruebas en dos tipos:

\begin{itemize}
 \item Pruebas sobre el funcionamiento de la interfaz.
 \item Pruebas de ejecución.
\end{itemize}


\section{Pruebas sobre el funcionamiento de la interfaz}
Tal y como indica el título de este apartado, se realizarán pruebas sobre los distintos elementos de la interfaz, las cuáles se irán describiendo
en las diferentes secciones presentadas a continuación.

\subsection{Visualización de opciones}
A continuación se detallarán las pruebas en relación a la correcta visualización de los distintos elementos de la interfaz, así como los errores 
encontrados y las soluciones aportadas. Se contemplan los siguientes casos:

\begin{itemize}
  \item Visualización de los distintos apartados y subapartados (si procede) de la barra de menú.
  \item Visualización de los distintos iconos que forman la barra de herramientas.
  \item Actualización de la información de la barra de estados.
  \item Visualización de información al dejar el cursor sobre un botón.
  \item Coloreado de palabras reservadas del lenguaje Sticky.
\end{itemize}

Se ha comprobado que, al hacer click en cualquier elemento principal del menú, se muestra el listado de acciones disponibles para cada sección.
Además, al pasar el ratón por dichas opciones, se puede observar como cambia la información de la barra de estado.

Por otra parte, podemos observar como aparece cierta información al dejar el ratón sobre cualquier botón de la barra de herramientas, la cuál
es coherente con respecto a la imagen mostrada.

Finalmente, también se ha comprobado que al escribir palabras reservadas de nuestro lenguaje, como pueden ser "mientras", "var" o "mover", se
muestran en azul, color elegido para dicho fin.

\subsubsection*{Problemas encontrados}

\begin{enumerate}
  \item La ventana de la aplicación se salía de los márgenes dependiendo del equipo en el que se estuviese usando.
  \item Se han encontrado palabras reservadas del lenguaje que no se definían como tal, es decir, no se mostraban en azul.
\end{enumerate}

\subsubsection*{Soluciones aplicadas}

\begin{enumerate}
  \item Se ha reajustado la ventana a una resolución óptima.
  \item Se han enumerado el conjunto de palabras que no se mostraban correctamente y se ha modificado el código oportuno.
\end{enumerate}

\subsection{Ejecución de opciones}
En este apartado se han puesto a prueba los diferentes elementos de la barra de menú y los botones de la barra de herramientas. En esta ocasión
se han tenido en cuenta los siguientes casos:

\begin{itemize}
  \item Funcionamiento de botones de la barra de herramientas.
  \item Ejecución de las distintas opciones de la barra de menús.
  \item Comprobación de funcionamiento de los atajos de teclado.
\end{itemize}

En primer lugar, atendiendo a las opciones del menú podemos afirmar que funcionan a la perfección, tanto al hacer click en las mismas como al
usar los atajos de teclado proporcionados, los cuáles aparecen mostrados justo en el lado derecho de la opción. Por lo tando, si tomamos como ejemplo la opción 
de deshacer código, similar a la de cualquier editor, se ejecutará tanto al realizar la acción "Edición -> Deshacer" como al pulsar "Control+Z".
Además, se ha comprobado que los botones, uno a uno, realizan su cometido de forma correcta y rápida.

\subsubsection*{Problemas encontrados}

\begin{enumerate}
  \item Nos hemos encontrado con opciones, tanto el la barra de menús como de herramientas, que no realizaban las acciones que se les requería.
	Por ejemplo, al elegir un nivel de depuración 0, no mostraba únicamente los errores producidos, sino que también mostraba otra información
	adicional que correspondía a otro nivel.
  \item En ocasiones, al ejecutar codigo inestable, no se informaba del error en el resultado de la interpretación. Un ejemplo ha sido al intentar
	hacer una conversión de un String a un Double o Integer en el lenguaje Sticky.
  \item Botones que realizaban s función únicamente de forma parcial. Por ejemplo, el botón que permite interpretar el código sólo realizaba su
	tarea la primera vez que se presionaba.
\end{enumerate}

\subsubsection*{Soluciones aplicadas}

\begin{enumerate}
  \item Se ha localizado el origen del error y se ha vuelto a codificar. Finalmente, se comprobaba que realmente la acción requerida era la acción
	que se realizaba.
  \item En este caso se ha analizado el código y se han capturado las excepciones necesarias para que el error sea mostrado en el apartado de la 
	interfaz "Resultado interpretación".
  \item Tomando el ejemplo anterior, se comprobó la codificación y ahora la ventana de animación se reinicia al interpretar un nuevo codigo.
\end{enumerate}

\end{document}


