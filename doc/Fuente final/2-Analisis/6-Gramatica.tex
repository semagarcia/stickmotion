\chapter{Gramática}
Una gramática ("G") desde el punto de vista de un autómata, es un conjunto finito de reglas que describen toda la secuencia de
símbolos pertenecientes a un lenguaje específico L.\\

Una gramática es una estructura algebraica formada por cuatro elementos fundamentales: 
\begin{center}
   G = { N$_{T}$, T$_{T}$, S, P } 
\end{center}
donde:\\
\begin{itemize}
   \item \textbf{\textit{N$_{T}$}} es el conjunto de símbolos no terminales.
   \item \textbf{\textit{V$_{T}$}} es el conjunto de símbolos terminales. 
   \item \textbf{\textit{S}} es el símbolo inicial de la gramática. 
   \item \textbf{\textit{P}} es el conjunto de reglas de producción. 
\end{itemize}

   \section{Símbolos terminales}
   Lo símbolos terminales (V$_{T}$) son aquellos no derivables. Los símbolos terminales se definen en el analizador léxico, siendo en este
   caso los definidos en la zona de tokens, los operadores definidos en la zona de reglas, letras, dígitos, cadenas, llaves, paréntesis y otros.\\
      \begin{minipage}{0.5\textwidth}
         % Con mathescape=false haremos que se obvie el signo del dólar
         \begin{lstlisting}[mathescape=false]
         ENTERO -> "int"
         REAL -> "real"
         VAR -> "var"
         SUP -> "sup"
         B_FOR -> "para"
         B_WHILE -> "mientras"
         IF -> "si"
         ELSE -> "sino"
         SWITCH -> "opcion"
         CASE -> "caso"
         END_CASE -> "fincaso"
         DEFAULT -> "defecto"
         GIRAR -> "girar";
         FLEXIONAR -> "flexionar"
         TIEMPO -> "tiempo";
         AVANZAR -> "avanzar"
         MOVER -> "mover"
         ESTABLECE -> "establece"
         AVANZA -> "avanza"
         CABEZA -> "CABEZA"
         BRAZO -> "BRAZO"
         PIERNA -> "PIERNA"
         STICKMAN -> "STICKMAN"
         IZQ -> "IZQ"
         DER -> "DER"
         IMPRIMIR -> "mostrar"
         PI -> "PI"
         NL -> ('\n' | ("\r\n")
         COM_LINEA -> "//" (~('\n'|'\r'))*
         FIN_INSTRUCCION -> ';'
         FIN_INTERPRETE -> '$'
         \end{lstlisting} 
      \end{minipage}\hfill
      \begin{minipage}{0.4\textwidth}
         \begin{lstlisting}[firstnumber=31]
         PAR_IZQ -> '('
         PAR_DER -> ')'
         SEPARA -> ','
         VERDADERO -> "VERDAD"
         FALSO -> "FALSO"
         DOBLE_PUNTO -> ':'
         LLAVE_IZQ -> '{'
         LLAVE_DER -> '}'
         OP_SUM -> '+'
         OP_RES -> '-'
         OP_MUL -> '*'
         OP_DIV -> '/'
         OP_INC -> "++"
         OP_DEC -> "--"
         OP_MOD -> '%'
         OP_POT -> '^'
         OP_RAIZ -> "raiz"
         OP_ASIG -> '='
         OP_MAYOR -> '>'
         OP_MENOR -> '<'
         OP_IG -> "=="
         OP_DIST: "!="
         OP_OX -> "OX"
         OP_NO -> "NO" 
         OP_Y -> 'Y'
         OP_O -> 'O'
         LETRA -> 'a'..'z'|'A'..'Z'
         DIGITO -> '0'..'9'  
         CADENA -> '"' ~('\\') | "\\\")* '"'
         \end{lstlisting} % 
      \end{minipage}


   \section{Símbolos no terminales}
   Los símbolos no terminales (V$_{N}$) son aquellos que se pueden derivar mediante reglas, hasta obtener un conjunto de símbolos terminales.
   En nuestro intérprete, definimos símbolos no terminales tanto en el analizador léxico como en el sintáctico.\\

   En el analizador léxico se definen los siguientes símbolos no terminales:\\
      \begin{lstlisting}
         BLANCO
         COMENTARIO
         COM_BLOQUE
         IDENT
         NUMERO
      \end{lstlisting}
   Pero la gran mayoría de símbolos no terminales se implementan en el análizador sintático. Como ya se ha comentado anteriormente, el analizador
   sintáctico ha sido implementado en el archivo \textit{anasint.g} mediante la clase \textbf{\textit{Anasint}} que hereda de la clase
   \textbf{\textit{Parser}} de ANTLR.\\

   El analizador sintáctico hará uso en sus reglas de símbolos terminales y no terminales. Los símbolos terminales son los reconocidos por el
   analizador léxico, que son comunicados al analizador sintáctico para ejecutar sus reglas. Los símbolos no terminales son aquellos que componen
   y etiquetan las reglas del analizador sintáctico.\\

   El analizador sintáctico se encargará de implementar las acciones sintácticas del intérprete. A continuación, en los sucesivos puntos, se
   describirán las distintas funciones empleadas por el analizador. En el analizador sintáctico se definen los siguientes símbolos no terminales:\\
      \begin{lstlisting}
         instrucciones
         sentencia
         simple
         bucle
         declaracion
         asignacion
         expr_or
         expr_xor
         expr_and
         expr_not
         expr_relacional
         expr_aritmetica
         expr_mod
         expr
         expr_mult
         expr_raiz
         expr_base
         expr_incremento
         evaluar_expr
         sentencia_if
         sentencia_switch
         casos_switch
         eliminar_var
         funcion_sticky
         f_tiempo
         f_mover
         f_flexionar
         f_girar
         fin_interprete
         sentencia_while
         sentencia_for
         imprimir
         impr_base
      \end{lstlisting}


   \section{Reglas de producción}
   Las reglas de producción se encargan de derivar los símbolos no terminales. Se muestran a continuación las reglas de producción de la
   gramática desarrollada:
      % Haciendo escapechar=\$ hacemos que se interpreten los signos dolar como código Látex
       \begin{lstlisting}[escapechar=\$]
         BLANCO $\rightarrow$ (' '|'\t'|NL)

         COMENTARIO $\rightarrow$ (COM_LINEA|COM_BLOQUE)

         COM_BLOQUE $\rightarrow$ "/*" ~('\n'|'\r')|NL)* "*/"

         IDENT $\rightarrow$ (LETRA|'_') (LETRA|'_'|DIGITO)*

         NUMERO $\rightarrow$ ((DIGITO )+ '.' ) => (DIGITO)+ '.' (DIGITO)* | (DIGITO)+

         instrucciones $\rightarrow$ (sentencia)* FIN_INSTRUCCION

         sentencia $\rightarrow$ (simple FIN_INSTRUCCION) | bucle

         simple $\rightarrow$ declaracion | asignacion | eliminar_var | funcion_sticky | imprimir | expr_incremento
         
         bucle $\rightarrow$ sentencia_if | sentencia_while | sentencia_for | sentencia_switch;

         declaracion $\rightarrow$ (VAR IDENT | VAR IDENT OP_ASIG | 
          VAR IDENT (SEPARA IDENT) *)
    
         asignacion $\rightarrow$ IDENT OP_ASIG expr_or
         
         expr_aritmetica $\rightarrow$ expr_mod (OP_MOD expr_mod)*
         
         expr_mod $\rightarrow$ expr (OP_SUM expr | OP_RES expr)*
         
         expr $\rightarrow$ expr_mult (OP_MUL expr_mult | OP_DIV expr_mult)*
         
         expr_mult $\rightarrow$ expr_raiz (OP_POT expr_raiz)*
         
         expr_raiz $\rightarrow$ expr_base (OP_RAIZ expr_base)*
         
         expr_base $\rightarrow$ ENTERO | REAL | VERDADERO | FALSO | CADENA | PI | 
               IDENT | expr_incremento | 
               (PAR_IZQ expr_aritmetica PAR_DER) | 
               (PAR_IZQ expr_or PAR_DER)
         
         expr_incremento $\rightarrow$ (IDENT OP_INC) | (IDENT OP_DEC)
         
         expr_or $\rightarrow$ expr_xor (OP_O expr_xor)*
         
         expr_xor $\rightarrow$ expr_and (OP_OX expr_and)*
         
         expr_and $\rightarrow$ expr_not (OP_Y expr_not)*
         
         expr_not $\rightarrow$  expr_relacional (OP_NO expr_relacional)*
         
         expr_relacional $\rightarrow$ expr_aritmetica 
               (OP_MAYOR expr_aritmetica |                    OP_MENOR expr_aritmetica |
                  OP_IG expr_aritmetica  |
                  OP_DIST expr_aritmetica )
         
         evaluar_expr $\rightarrow$  expr_or
         
         sentencia_if $\rightarrow$  (IF PAR_IZQ evaluar_expr PAR_DER LLAVE_IZQ) 
                  ((sentencia)*|$\varepsilon$) LLAVE_DER 
      
                  ((ELSE LLAVE_IZQ
                  ((sentencia)*|$\varepsilon$) LLAVE_DER)
               |
                  (ELSE ~LLAVE_IZQ
                  (sentencia | $\varepsilon$))
                  )?
               |
               (IF PAR_IZQ evaluar_expr PAR_DER)
                  (sentencia | $\varepsilon$)
         
                  ((ELSE ~LLAVE_IZQ
                     (sentencia | $\varepsilon$))          
                  |
                     (ELSE LLAVE_IZQ
                        ((sentencia)* | $\varepsilon$) 
                     LLAVE_DER)
                        )?
    
         sentencia_switch $\rightarrow$ SWITCH PAR_IZQ expr_or PAR_DER                     LLAVE_IZQ (casos_switch)*
                  (DEFAULT DOBLE_PUNTO LLAVE_IZQ
                  ((sentencia)* | $\varepsilon$ ) LLAVE_DER 
                  END_CASE FIN_INSTRUCCION)? LLAVE_DER

            casos_switch $\rightarrow$ CASE expr_or DOBLE_PUNTO LLAVE_IZQ 
                  ((sentencia)* | $\varepsilon$ ) LLAVE_DER
                  END_CASE FIN_INSTRUCCION

         eliminar_var $\rightarrow$ SUP IDENT
         
         funcion_sticky $\rightarrow$ f_tiempo | f_mover | f_flexionar | f_girar
         
         f_tiempo $\rightarrow$ TIEMPO (ESTABLECE | AVANZA) 
            PAR_IZQ expr_aritmetica PAR_DER
         
         f_mover $\rightarrow$ MOVER STICKMAN PAR_IZQ
            expr_aritmetica SEPARA
            expr_aritmetica SEPARA
            expr_aritmetica SEPARA
            expr_aritmetica PAR_DER 
         
         f_flexionar $\rightarrow$ FLEXIONAR (brazo:BRAZO | PIERNA) (DER | IZQ) 
               PAR_IZQ expr_aritmetica 
               SEPARA expr_aritmetica 
               PAR_DER
         
         f_girar $\rightarrow$ GIRAR (STICKMAN | CABEZA | BRAZO | PIERNA) (DER | IZQ)?
               PAR_IZQ expr_aritmetica 
               SEPARA expr_aritmetica 
               SEPARA expr_aritmetica 
               PAR_DER
         
         fin_interprete $\rightarrow$ FIN_INTERPRETE
         
         sentencia_while $\rightarrow$ ((B_WHILE PAR_IZQ evaluar_expr PAR_DER LLAVE_IZQ) 
                  ((sentencia)* | $\varepsilon$)  LLAVE_DER)
                  |
               ((B_WHILE PAR_IZQ evaluar_expr PAR_DER)
                  (sentencia | $\varepsilon$))

         sentencia_for $\rightarrow$ (B_FOR 
               PAR_IZQ 
               IDENT 
               FIN_INSTRUCCION evaluar_expr 
               FIN_INSTRUCCION ENTERO 
               PAR_DER 
               LLAVE_IZQ
               ((sentencia)* | $\varepsilon$ ) 
               LLAVE_DER)
               |  
               (B_FOR 
               PAR_IZQ IDENT 
               FIN_INSTRUCCION evaluar_expr 
               FIN_INSTRUCCION ENTERO 
               PAR_DER
               (sentencia | $\varepsilon$) )

         imprimir $\rightarrow$ IMPRIMIR PAR_IZQ impr_base
            (OP_SUM impr_base)* PAR_DER
         
         impr_base $\rightarrow$ (CADENA | IDENT)
       \end{lstlisting}


   \section{Acciones semánticas}
   Las acciones semánticas correspondientes a la derivación de cada símbolo no terminal, así como la implementación de todas las
   funciones que reconocerá el intérprete, serán especificadas en el capítulo 9, donde se describe el analizador sintáctico y semántico. 















