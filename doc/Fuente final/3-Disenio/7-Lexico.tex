\chapter{Analizador léxico}
En ANTLR no se utilizan expresiones regulares y autómatas finitos para especificar e implementar, respectivamente, analizadores
léxicos. En su lugar se aprovechan la misma notación y el mismo tipo de reconocedores utilizados en la generación de analizadores
sintácticos: gramáticas EBNF y reconocedores recursivos descendentes. Con esta decisión se gana en homogeneidad a nivel de notación,
aunque en algunas ocasiones se echan de menos algunas facilidades propias de las expresiones regulares, fundamentalmente el hecho de 
no tener que preocuparse de que los patrones léxicos tengan prefijos comunes, problema que en los reconocedores descendentes está 
siempre presente. \\

El analizador léxico se compone de 4 partes: 
\begin{itemize}
   \item \textbf{\textit{Zona de cabecera}}: indica los paquetes que se van a necesitar para contruir el archivo y de los cuáles son
         dependientes. 
   \item \textbf{\textit{Zona de opciones}}: donde se activan algunas características del analizador léxico. 
   \item \textbf{\textit{Zona de tokens}}: donde se definirán las palabras reservadas del lenguaje así como también los tokens 
         utilizados para ciertas reglas predictivas. 
   \item \textbf{\textit{Zona de reglas}}: donde se definen los tokens que reconocerá el intérprete.
\end{itemize}

A continuación abordaremos con mayor nivel de detalle estas secciones.\\

   \section{Zona de cabecera}
   \begin{lstlisting}
      header {
            package sticky;
      }
   \end{lstlisting}


   \section{Zona de opciones}
   En esta zona se presentan una serie de opciones globales destinadas a particularizar el lexer.\\
   \begin{lstlisting}
      options {
         k=2;
         importVocab=Anasint;
         exportVocab=Analex;
      }
   \end{lstlisting}
   Se han utilizado las siguientes opciones fundamentales: 
   \begin{itemize}
      \item \textbf{\textit{k}}: número de token \textit{lookahead}. Alterar su número permite decidir alternativas con más información (más tokens).
      \item \textbf{\textit{importVocab}}: indica que va a importar el vocabulario y tokens definidos en el analizador \textit{Anasint}.
      \item \textbf{\textit{exportVocab}}: indica que todos los símbolos y tokens definidos aquí van a ser exportados. 
   \end{itemize}


   \section{Zona de tokens}
   ANTLR permite especificar palabras reservadas en la sección tokens. Se han definido los siguientes tokens:
   \begin{lstlisting}
      tokens {
         ENTERO = "int";
         REAL = "real";
      
         // Variables
         VAR = "var";
         SUP = "sup";
         
         // Loops
         B_FOR = "para"; 
         B_WHILE = "mientras";
         
         // Conditional Structures
         IF = "si";
         ELSE = "sino";
         SWITCH = "opcion";
         CASE = "caso";
         END_CASE = "fincaso";
         DEFAULT = "defecto"; 
      
         // Stickman's movements
         GIRAR = "girar";
         FLEXIONAR = "flexionar";
         TIEMPO = "tiempo";
         AVANZAR = "avanzar";
         MOVER = "mover";
         ESTABLECE = "establece";
         AVANZA = "avanza";
         CABEZA = "CABEZA";
         BRAZO = "BRAZO";
         PIERNA = "PIERNA";
         STICKMAN = "STICKMAN";
         IZQ = "IZQ";
         DER = "DER";
         
         // Display print
         IMPRIMIR="mostrar";
         
         // Others variables
         PI = "PI";
      }
   \end{lstlisting}


   \section{Zona de reglas}
   Las reglas léxicas constituyen la parte esencial del analizador léxico y están destinadas a definir tokens como secuencias de
   caracteres desde el flujo de entrada.\\ 
  
   Las reglas léxicas presentan la forma token: patrón y se implementan en el analizador resultante como métodos. Se han definido
   las siguientes reglas: 
   \begin{lstlisting}[mathescape=false]
      protected NL: ('\n' |
         ("\r\n")=>
            "\r\n" |
            '\r')
         {newline();};

      BLANCO: (' '|'\t'|NL) {$setType(Token.SKIP);};

      // Comments
      COMENTARIO: (COM_LINEA|COM_BLOQUE) {$setType(Token.SKIP);};
      COM_LINEA: "//" (~('\n'|'\r'))*;
      COM_BLOQUE: "/*" (options {greedy=false;}: ~('\n'|'\r')|NL)* "*/";

      // Instruction end
      FIN_INSTRUCCION: ';';
      FIN_INTERPRETE: '$';
      
      PAR_IZQ: '(';
      PAR_DER: ')';
      SEPARA: ',';
      VERDADERO : "VERDAD";
      FALSO : "FALSO";

      DOBLE_PUNTO: ':'; 
      
      LLAVE_IZQ: '{';
      LLAVE_DER: '}';
      
      // Arithmetic operators
      OP_SUM: '+';
      OP_RES: '-';
      OP_MUL: '*';
      OP_DIV: '/';
      OP_INC: "++";
      OP_DEC: "--";
      OP_MOD: '%';
      OP_POT: '^';
      OP_RAIZ: "raiz";
      
      // Assignment Operators
      OP_ASIG: '=';
         
      // Conditionals
      OP_MAYOR: '>';
      OP_MENOR: '<';
      OP_IG: "==";
      OP_DIST: "!=";
      OP_OX : "OX";
      OP_NO : "NO"; 
      OP_Y : 'Y';
      OP_O : 'O';

      // Identifiers
      protected LETRA: 'a'..'z'|'A'..'Z';
      protected DIGITO: '0'..'9';
      IDENT: (LETRA|'_') (LETRA|'_'|DIGITO)*;

      // Numbers
      NUMERO: ( (DIGITO )+ '.' ) => (DIGITO )+ '.' ( DIGITO )* { $setType (REAL); } | ( DIGITO )+ { $setType (ENTERO);};
            
      // Strings
      CADENA: '"' (options {greedy=false;}: ~('\\') | "\\\"")* '"';
   \end{lstlisting}

