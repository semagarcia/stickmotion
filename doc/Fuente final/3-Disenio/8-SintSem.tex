\chapter{Analizadores sintáctico y semántico}
El analizador sintáctico generado se implementará a través de la clase \textit{Anasint} que será una subclase de \textit{Parser}
que es la que implementa los aspectos genéricos de los reconocedores sintácticos. La clase \textit{Anasint} ofrece un método por cada
símbolo terminal de la gramática, dichos métodos serán los encargados de reconocer el lenguaje asociado a los símbolos correspondientes.\\

   % Se puede añadir un caption sin label: [title=Periquito de los palotes]
   % O con label: [caption={Useless code},label=useless]
   \begin{lstlisting}
      options {
         k=2; 
      }      
      {
         SymbolTable tablaSimbolos = new SymbolTable();
         double const_pi = Math.PI;
         // Exception function
         void mostrarExcepcion(RecognitionException re)
         {
            Processor.println(0,"En linea " + re.getLine() + " : " + re.getMessage());
            reportError(re);
            try {
               consume(); // Consume the token problem
               consumeUntil(FIN_INSTRUCCION);
            } catch (Exception e) {
            }
         }
      }
    \end{lstlisting}
   Las acciones semánticas, se incrustan en el código, en este caso, en el lenguaje de programación Java. A continuación se explican
   detalladamente, una a una, las consideraciones sintácticas y semánticas para cada una de las funciones del lenguaje.


   \section{Instrucciones}
   \begin{lstlisting}
      instrucciones 
      {
         Object x = null;
         Processor.println(-1, "...INICIANDO STICKY...");   
      }
      : (sentencia)* fin_interprete;
   \end{lstlisting}

   \section{Sentencia}
   Esta regla es la raiz del lenguaje, es la producción S de la gramática. Admite 0 o muchas sentencias. Las instrucciones van seguidas
   del carácter de fin de interpretación.
   \begin{lstlisting}
      sentencia: (simple FIN_INSTRUCCION) | bucle; 
      // Either one simple sentence with ; or loop, that no needs it;
      exception
         catch [RecognitionException re] {
            mostrarExcepcion(re);
         }
   \end{lstlisting}

   \section{Simple}
   Una instrucción simple puede ser una declaración, una asignación, una instrucción sticky, el incremento o decremento del valor 
   asignado a un identificador, la eliminación de una variable en la tabla de símbolos o la impresión de un valor por pantalla.
   \begin{lstlisting}
      simple {String valor; Object valor2;}:  
            declaracion    | 
            asignacion     | 
            eliminar_var   | 
            funcion_sticky | 
            valor = imprimir { Processor.println(-1, valor); } | 
         valor2 = expr_incremento;
   \end{lstlisting}

   \section{Bucle}
   Los bucles y sentencias condicionales aceptadas son:
   \begin{lstlisting}
      bucle: sentencia_if | sentencia_while | sentencia_for | sentencia_switch;
      exception
         catch [RecognitionException re] {
            mostrarExcepcion(re);
         }
   \end{lstlisting}

   \section{Declaración}
   Esta regla permite la declaración de variables de tres formas claramente diferenciadas:
   \begin{enumerate}
      \item Se declara una variable sin inicializarse.
      \item Se declara una variable y si se inicializa.
      \item Se declara más de una variable.
   \end{enumerate}
   ¡Ojo! NO se permite la inicialización de variables si se declara más de una.
   \begin{lstlisting}
      declaracion {String mensaje;Object x = null; ArrayList lista = new ArrayList();}:
      ( // Option 1
         (VAR IDENT FIN_INSTRUCCION) => VAR i1:IDENT
         { 
            //... introducción en la tabla de símbolos y control de errores ...
         }
         
         // Option 2
         |(VAR IDENT OP_ASIG expr_or) =>VAR i3:IDENT OP_ASIG (x=expr_or) 
            {     
            // ... introducción en la tabla de símbolos y control de errores
            }
         // Option 3
         |(VAR IDENT (SEPARA IDENT)*) => VAR i4:IDENT (SEPARA i3_alt:IDENT
                        {lista.add(i3_alt);} )* 
               {
               //... inserción en la tabla de símbolos de cada
               // identificador encontrado y control de errores ...
               }
               );
         exception
            catch [RecognitionException re] {
               mostrarExcepcion(re);
            }
   \end{lstlisting} 
   Una excepción se disparará si se encuentra algún token que no proceda.\\
      
   \section{Asignación}
   Esta regla se ocupa de gestionar las asignaciones a realizar sobre variables ya declaradas, mostrando un mensaje de error 
   si la asignación no se ha realizado porque la variable no existe en la tabla de símbolos.
   \begin{lstlisting}
      asignacion { Object respuesta; Object respuesta2;}:
      i2:IDENT OP_ASIG (respuesta = expr_or)
      {   
            if(tablaSimbolos.set(i2,respuesta)) 
            Processor.println(1, "Linea "+i2.getLine()+": Asignacion a la variable \""+i2.getText()+"\": "+respuesta);
            else 
              Processor.println(0, "Linea "+i2.getLine()+": Asignacion no realizada, no existe la variable \""+i2.getText()+"\"");
      };
      exception
         catch [RecognitionException re] {
            mostrarExcepcion(re);
         }
   \end{lstlisting}

   \section{expr\_or}
   Esta regla recibe un valor desde \textit{expr\_xor}, y si encuentra \textit{OP\_O} y otro valor, realiza la disyunción, 
   comprobando si la operación es posible porque es acorde a los tipos de datos recibidos desde las reglas hijas. (Para realizar
   la disyunción ambos tipos de datos deben ser booleanos). En caso de no poderse realizar la operación muestra mensaje de error.\\

   Además, en caso de producirse un error de tipo sintático, una excepción se dispara, informando y gestionando del error, 
   consumiendo todos los tokens hasta el siguiente fin de instrucción (método de pánico).\\
   \begin{lstlisting}
      expr_or returns [Object resultado = null] 
      {Object exp1 = null; Object exp2 = null; Object exp3 = null;}: 
      exp1= expr_xor
      {
         resultado = exp1;
      }
      ( linea:OP_O exp2 = expr_xor
         {
         //...Comprueba los tipos de datos, y en caso de que proceda muestra error el resultado...
         } 
      )*;
      exception
         catch [RecognitionException re] {
            mostrarExcepcion(re);
         }
   \end{lstlisting}

   \section{expr\_xor}
   Esta regla recibe un valor desde \textit{expr\_and}, y si encuentra \textit{OP\_OX} y otro valor, realiza la disyunción 
   exclusiva, comprobando si la operación es posible porque es acorde a los tipos de datos recibidos desde las reglas hijas. 
   (Para realizar la disyunción exclusiva ambos tipos de datos deben ser booleanos).\\
  
   Si la operación lógica no se puede realizar, se muestra un mensaje de error.
   \begin{lstlisting}
      expr_xor returns [Object resultado = null]
      {Object exp1 = null; Object exp2 = null; Object exp3 = null;}: 
      exp1= expr_and
      {
         resultado = exp1;
      }
      ( linea:OP_OX exp3 = expr_and
         {
         //...Comprueba los tipos, y en caso de que proceda muestra error resultado...
         } 
      )*;
   \end{lstlisting}

   \section{expr\_and}
   Esta regla recibe un valor desde \textit{expr\_not}, y si encuentra \textit{OP\_Y} y otro valor, realiza la conjunción, 
   comprobando si la operación es posible porque es acorde a los tipos de datos recibidos desde las reglas hijas. (Para 
   realizar la conjunción ambos tipos de datos deben ser booleanos).\\
  
   Si las operación no se puede realizar se muestra mensaje de error.
   \begin{lstlisting}
      expr_and returns [Object resultado = null]
      {Object exp1 = null ; Object exp2 = null;}: 

      exp1=expr_not {resultado = exp1;}
      ( linea:OP_Y exp2 = expr_not
         {
         //...Comprueba los tipos, y en caso de que proceda muestra error o el resultado...
         }
      )*;
   \end{lstlisting}

   \section{expr\_not}
   Esta regla responde a la negación de expresiones booleanas. Muestra error si la operación no se puede realizar.
   \begin{lstlisting}
      expr_not returns [Object resultado = null]
      {Object exp1 = null ; Object exp2 = null;}: 

      (expr_relacional) => exp1=expr_relacional { resultado = exp1;}
      |
      ( linea:OP_NO exp2 = expr_relacional
         {
         //...Comprueba los tipos, y en caso de que proceda muestra error o
         //el resultado, para evitar negar una variable que no sea booleana...
         }
      )*;
   \end{lstlisting}

   \section{expr\_relacional}
   Esta regla es la encargada de evaluar las expresiones relacionales. La regla recibe un valor (independientemente del
   tipo), y lo transmite hacia reglas padre.\\

   Si además recibe un operador relacional (\textit{OP\_MAYOR, OP\_MENOR, OP\_IG, OP\_DIST}) y un valor, comprueba que ambos 
   sean numéricos (realizando las conversiones necesarias entre tipos Java), y obtiene un booleano como resultado de la expresión 
   relacional, transmitiéndolo hacia la raíz en el árbol gramatical.\\

   Si se intenta realizar operaciones relacionales con cadenas de caracteres se muestra un mensaje de error.
   \begin{lstlisting}
      expr_relacional returns [Object respuesta = null] {Object e1; Object e2;}:
      e1 = expr_aritmetica {
         respuesta = e1;
      }
      (linea:OP_MAYOR e2 = expr_aritmetica
      {
         if(e1 instanceof Integer && e2 instanceof Integer)
         {
            int num1 = new Integer(e1.toString()).intValue();
            int num2 = new Integer(e2.toString()).intValue();
            respuesta = new Boolean(num1>num2);
         }
         
         if(e1 instanceof Integer && e2 instanceof Double)
         {
            int num1 = new Integer(e1.toString()).intValue();
            double num2 = new Double(e2.toString()).doubleValue();
            respuesta = new Boolean(num1>num2);
         }
         
         if(e1 instanceof Double && e2 instanceof Double)
         {
            double num1 = new Double(e1.toString()).doubleValue();
            int num2 = new Integer(e2.toString()).intValue();
            respuesta = new Boolean(num1>num2);
         }
         
         if(e1 instanceof Double && e2 instanceof Double)
         {
            double num1 = new Double(e1.toString()).doubleValue();
            double num2 = new Double(e2.toString()).doubleValue();
            respuesta = new Boolean(num1>num2);
         }
         
         
      if(((e1 instanceof String == true) && (e2 instanceof String == false))
      ||((e1 instanceof String == false)&&(e2 instanceof String == true)))
         {
            Processor.println(0,"Linea "+linea.getLine()+": No se puede 
            realizar expresiones relacionales con cadenas");
            respuesta = new Boolean(false);
         }
         
            
      }
      |linea2:OP_MENOR e2 = expr_aritmetica
      {
      // Igual que para el OP\_MAYOR
      } 
      |linea3:OP_IG e2 = expr_aritmetica
      {
      // Igual que para el OP\_MAYOR
      }
      |linea4:OP_DIST e2 = expr_aritmetica
      {
      // Igual que para el OP\_MAYOR
      }
      )*;
   \end{lstlisting}

   \section{expr\_aritmetica}
   Esta regla es la encargada de evaluar las expresiones aritmética. La regla recibe un valor (independientemente del tipo),
   y lo transmite hacia reglas padre.\\

   Si además recibe el operador de módulo (\textit{OP\_MOD}) y un valor, comprueba que ambos sean numéricos (realizando las conversiones
   necesarias entre tipos Java), y obtiene un número (real o entero) como resultado de la expresión aritmética, transmitiéndolo
   hacia la raíz en el árbol gramatical.\\

   En caso de intentar realizar operaciones aritméticas con cadenas de caracteres o expresiones booleanos se muestra un mensaje de 
   error. Además, en caso de error de tipo sintáctico, una excepción se dispara capturando y gestionando dicho error.
   \begin{lstlisting}
      expr_aritmetica returns[Object resultado = null]
      {Object e1 = null; Object e2 = null; Object e3 = null;}:
      e1 = expr_mod
      {
         // This is in case of var =- 2; The char - at the top makes 
         // the first operand null and the second 2
         if(e1 == null)
            e1 = 0;
            
         resultado = e1;
      }
      ( linea1:OP_MOD e2 = expr_mod
      {
         if(e1 instanceof Integer && e2 instanceof Integer)
         {
            int valor1 = new Integer(e1.toString()).intValue();
            int valor2 = new Integer(e2.toString()).intValue();
            
            resultado = new Integer(valor1%valor2);
         }
         
         if((e1 instanceof Double || e2 instanceof Double) &&
            !(e1 instanceof String || e2 instanceof String || 
            e1 instanceof Boolean || e2 instanceof Boolean))
         {
            double valor1 = new Double(e1.toString()).doubleValue();
            double valor2 = new Double(e2.toString()).doubleValue();
            
            resultado = new Double(valor1 % valor2);
         }
         
         if(e1 instanceof String || e2 instanceof String)
         {
            Processor.println(0,"Linea "+linea1.getLine()
            +": No se pueden realizar operaciones aritmeticas 
            con cadenas de caracteres");
         }
         
         if(e1 instanceof Boolean || e2 instanceof Boolean)
         {
            Processor.println(0,"Linea "+linea1.getLine() +": No se pueden realizar operaciones aritmeticas con valores booleanos");
         }
         
      })*;
      exception
         catch [RecognitionException re] {
            mostrarExcepcion(re);
         }
   \end{lstlisting}

   \section{expr\_mod}
   Esta regla es la encargada de evaluar las expresiones módulo. La regla recibe un valor (independientemente del tipo),
   y lo transmite hacia reglas padre. \\

   Si además recibe los operadores aritméticos (\textit{OP\_SUM} u \textit{OP\_RES}) y un valor, comprueba que ambos sean numéricos 
   (realizando las conversiones necesarias entre tipos Java), y obtiene un número (real o entero) como resultado de la expresión aritmética,
   transmitiéndolo hacia la raíz en el árbol gramatical.\\

   Si se intenta realizar sumas o restas con valores no numéricos se muestra un mensaje de error en el interfaz del intérprete.
   \begin{lstlisting}
      expr_mod returns [Object resultado = null]
      {Object e1; Object e2; Object e3; Object e4;}: 
      e1 = expr
      {
         // This is in case of var =- 2; The char - at the top makes 
         // the first operand null and the second 2
         if(e1 == null)
            e1 = 0;
         resultado = e1;
      }
      ( linea1:OP_SUM e2 = expr
      {
         if(e1 instanceof Integer && e2 instanceof Integer) {
            int valor1 = new Integer(e1.toString()).intValue();
            int valor2 = new Integer(e2.toString()).intValue();
            
            resultado = new Integer(valor1+valor2);
            
            e1=resultado;
         } else {
            if(e1 instanceof String || e2 instanceof String) {
               if(e1 instanceof String && e2 instanceof String) {
                  String valor1 = new String(e1.toString());
                  String valor2 = new String(e2.toString());
                  resultado = new String(valor1+valor2);
               } else {
                  Processor.println(0,"Linea "+linea1.getLine()+":
                     No se pueden realizar operaciones aritmeticas
                     con cadenas de caracteres");
               }
            } else {
               if(e1 instanceof Boolean || e2 instanceof Boolean)
                  Processor.println(0,"Linea "+linea1.getLine()+":
                     No se pueden realizar operaciones aritmeticas
                     con valores booleanos");
               else {
                  double valor1 = new Double(e1.toString()).doubleValue();
                  double valor2 = new Double(e2.toString()).doubleValue();
                  resultado = new Double(valor1+valor2);
                  e1 = resultado;
               }
            }
         } 
      }
      | linea2:OP_RES e3 = expr
      {
      // Se comprueban los tipos de datos como en la suma para realizar la 
      // operación de sustracción.
      }
      )*;
   \end{lstlisting}

   \section{expr}
   Esta regla es la encargada de evaluar las expresiones de multiplicación o división. La regla recibe un valor 
   (independientemente del tipo), y lo transmite hacia reglas padre. \\
   Si además recibe los operadores aritméticos (\textit{OP\_MUL} u \textit{OP\_DIV}) y otro valor, comprueba que ambos sean 
   numéricos (realizando las conversiones necesarias entre tipos Java), y obtiene un número (real o entero) como resultado de la
   expresión aritmética, transmitiéndolo hacia la raíz en el árbol gramatical.\\
   
   Si se intentan realizar multiplicaciones o divisiones con valores no numéricos se muestra un mensaje de error en el 
   interfaz del intérprete.
   \begin{lstlisting}
      expr returns [Object resultado = null]
      {Object e1; Object e2; Object e3;}: 
      
      e1 = expr_mult {
         // This is in case of var =- 2; The char - at the top makes 
         // the first operand null and the second 2
         if(e1 == null)
            e1 = 0;
            
         resultado = e1;
         }

      ( linea1:OP_MUL e2 = expr_mult
         {            
         if(e1 instanceof Integer && e2 instanceof Integer) {
            int valor1 = new Integer(e1.toString()).intValue();
            int valor2 = new Integer(e2.toString()).intValue();
            
            resultado = new Integer(valor1*valor2);
            
            e1=resultado;
         }
            
         if((e1 instanceof Double || e2 instanceof Double) && 
            !(e1 instanceof String || e2 instanceof String || 
            e1 instanceof Boolean || e2 instanceof Boolean)) {
            double valor1 = new Double(e1.toString()).doubleValue();
            double valor2 = new Double(e2.toString()).doubleValue();
            
            resultado = new Double(valor1*valor2);
            
            e1=resultado;
         }
            
         if(e1 instanceof String || e2 instanceof String) {
            Processor.println(0,"Linea "+linea1.getLine()+": No se
               pueden realizar operaciones aritmeticas con cadenas de
               caracteres");
         }
            
         if(e1 instanceof Boolean || e2 instanceof Boolean) {
            Processor.println(0,"Linea "+linea1.getLine()+": No se
               pueden realizar operaciones aritmeticas con valores
               booleanos");
            }
         }
         | linea2:OP_DIV e3 = expr_mult
         {
            // Comprueba los tipos de datos como en la multiplicación para realizar 
            // la división 
         }
         )*;
   \end{lstlisting}

   \section{expr\_mult}
   Esta regla es la encargada de evaluar la expresión de potencia. La regla recibe un valor (independientemente del tipo), y
   lo transmite hacia reglas padre.\\

   Si además recibe el operador (\textit{OP\_POT}) y otro valor, comprueba que ambos sean numéricos (realizando las conversiones necesarias
   entre tipos Java), y obtiene un número (real o entero) como resultado de la expresión, transmitiéndolo hacia la raíz en el árbol
   gramatical. 
   \begin{lstlisting}
      expr_mult returns [Object resultado = null] {Object e1; Object e2;}:
      e1 = expr_raiz
      {
         if(e1 == null)
            e1 = 0;
            
         resultado = e1; 
      }
      (linea1:OP_POT e2 = expr_raiz
      {
         // Comprueba los tipos de e1 y e2 para realizar la potencia 
      }
      )*;
   \end{lstlisting}

   \section{expr\_raiz}
   Esta regla es la encargada de evaluar la expresión de raiz cuadrada. La regla recibe un valor (independientemente del tipo), 
   y lo transmite hacia reglas padre. \\

   Si además recibe el operador (\textit{OP\_RAIZ}) y otro valor, comprueba que ambos sean numéricos (realizando las conversiones necesarias
   entre tipos Java), y obtiene un número (real o entero) como resultado de la expresión raiz, transmitiéndolo hacia la raíz en el
   árbol gramatical. 
   \begin{lstlisting}
      expr_raiz returns [Object resultado = null]
      {Object exp1 = null ; Object exp2 = null;}: 

      exp1 = expr_base {
         // This is in case of var =- 2; The char - at the top makes 
         // the first operand null and the second 2
         if(exp1 == null)
            exp1 = 0; 
            resultado = exp1;
         }
      |
      ( linea:OP_RAIZ exp2 = expr_base
         {
         // Comprueba el tipo exp2 y en caso de ser numérico, realiza la raiz 
         // cuadrada del mismo. Si no muestra error.
         }
         )* ;
   \end{lstlisting}

   \section{expr\_base}
   La siguiente regla es la expresión base de la sección de la gramática descrita en el analizador sintáctico. Una expresión
   base puede ser un entero, un número real, un valor ``verdadero'', un valor ``falso'', una cadena, una constante \textit{PI}, un identificador
   correspondiente a una variable, una expresión de incremento o decremento, una expresión aritmética entre paréntesis o una 
   expresión OR (raiz de todas las operaciones booleanas, aritméticas, relacionales o aritméticas) entre paréntesis.\\

   Si se utiliza un identificador que no está declarado en la tabla de símbolos se muestra mensaje de error. Además, una excepción
   se dispara si se encuentra un error durante el análisis sintáctico, informando y resolviéndose este error, consumiendo todo 
   hasta el siguiente fin de instrucción (método de pánico).
   \begin{lstlisting}
      expr_base returns [Object resultado = null]: 
         n1:ENTERO {resultado = new Integer(n1.getText());}
         | n2:REAL {resultado = new Double(n2.getText());}
         | n3:VERDADERO {resultado = new Boolean(true);}
         | n4:FALSO {resultado = new Boolean(false);}
         | n5:CADENA {resultado = new String( n5.getText()); }
         | n6:PI { resultado = new Double(const_pi); 
               Processor.println(1,"PI declarada: "+ resultado); }
         | (IDENT) => id:IDENT
         {
      
         // Rescata el valor de la variable y lo devuelve. En caso de no 
         // existir muestra error
      
         }
         | resultado = expr_incremento
         | (PAR_IZQ expr_aritmetica PAR_DER) => PAR_IZQ (resultado =
                        expr_aritmetica) PAR_DER
         | (PAR_IZQ expr_or PAR_DER) => PAR_IZQ (resultado = expr_or) PAR_DER
         ;
      exception
         catch [RecognitionException re] {
            mostrarExcepcion(re);
         }
   \end{lstlisting}

   \section{expr\_incremento}
   Esta regla de la gramática es la responsable de reconocer las expresiones de incremento o decremento sobre los 
   identificadores declarados. \\

   Si el identificador referenciado no se corresponde con ninguna variable declarada en la tabla de símbolos se 
   muestra un mensaje de error, informando de este hecho.
   \begin{lstlisting}
      expr_incremento returns [Object resultado = null]:
      (IDENT OP_INC) => id2:IDENT OP_INC
         {
            if(tablaSimbolos.existeSimbolo(id2.getText()))
            {
            String contenido = tablaSimbolos.getContenidoSimbolo(id2.getText());
            
            if(contenido.compareTo("") == 0) {
            Processor.println(0,"Linea "+id2.getLine()+": 
               La variable "+id2.getText()+" no tiene asignado 
               ningun valor.");  
            } else {
            // If there is a string (if it takes " 'matches' returns false)
            if(contenido.matches("[0-9.]+|true|false")) {
               //If false o true is a boolean
               if(contenido.matches("true|false")) {
                  Processor.println(0,"Linea "+
                  id2.getLine()+": la variable "+
                  id2.getText()+" es un booleano y por 
                  tanto no se puede incrementar.");
               } // If takes only numeric values
                else if(contenido.matches("[0-9]+")) {
                  resultado = new
                  Integer(contenido.toString()).intValue()+1;
                  tablaSimbolos.set(id2,resultado);
               } // Else is a integer
                else {
                  resultado = new
                  Float(contenido.toString()).floatValue()+1;
                  tablaSimbolos.set(id2,resultado);
                }
               } else { //is a string
               Processor.println(0,"Linea "+id2.getLine()+":
                  la variable "+id2.getText()+" es una cadena y 
                  por tanto no se puede incrementar.");
               }
            }
         } else 
            Processor.println(0,"Linea "+id2.getLine()+":
            la variable "+id2.getText()+" no ha sido declarada.");
         }
         | (IDENT OP_DEC) => id3:IDENT OP_DEC
         {
         
         // Al igual que para el incremento, rescata la variable, la
         // incrementa, guarda su valor y lo devuelve. Si el identificador
         // no existe muestra error...    
         };
   \end{lstlisting}

   \section{sentencia\_if}
   La siguiente regla de la gramática implementa las sentencia ``if'', de las cuatro siguientes formas, según se usen con o sin corchetes: 
   \begin{enumerate}
      \item if de la forma: si (VERDAD) { sent = 1; } sino { sent = 2; var sentSent; }
      \item if de la forma: si (VERDAD) sent = 1; sino sent = 2;
      \item if de la forma: si (VERDAD) {sent = 1; } sino sent = 2;
      \item if de la forma: si (VERDAD) sent = 1; sino { sent = 2; }
   \end{enumerate}
   Una excepción se dispara si se produce un error en el análisis sintáctico, informando y gestionando el error, consumiendo todos los 
   tokens siguientes hasta el fin de instrucción.
   \begin{lstlisting}
      sentencia_if {Object o; boolean b = false;} : 
         (IF PAR_IZQ (evaluar_expr) PAR_DER LLAVE_IZQ) => //1
         IF id:PAR_IZQ (o=evaluar_expr) PAR_DER LLAVE_IZQ
            {
               if (o.getClass() == Boolean.class)
                  b = ((Boolean)o).booleanValue();
               else 
                  Processor.println(0, "Linea "+id.getLine()+":ERROR IF");
            }
            ({b==true}? (sentencia)* LLAVE_DER
            | {b==false}? (options{greedy=false;}:.)+ LLAVE_DER) 
            
            ((ELSE LLAVE_IZQ) => ELSE LLAVE_IZQ
            ({b==false}? (sentencia)* LLAVE_DER
            | {b==true}? (options{greedy=false;}:.)+ LLAVE_DER
            )
            |
            (ELSE ~LLAVE_IZQ) => ELSE               //3
            ({b==false}? sentencia
            | {b==true}? (options{greedy=false;}:.)+ FIN_INSTRUCCION
            )
            )?
            |
            (IF PAR_IZQ (evaluar_expr) PAR_DER) =>          //2
               id2: IF PAR_IZQ (o=evaluar_expr) PAR_DER 
            {
                     if (o.getClass() == Boolean.class)
                                 b = ((Boolean)o).booleanValue();
                     else Processor.println(0, "Linea "+id2.getLine()+":ERROR IF");
            }
            ({b==true}? sentencia 
            | {b==false}? (options{greedy=false;}:.)+ FIN_INSTRUCCION)
            
            (
            (ELSE ~LLAVE_IZQ) => ELSE
            ({b==false}? sentencia
            | {b==true}? (options{greedy=false;}:.)+ FIN_INSTRUCCION  
            )
            |
            (ELSE LLAVE_IZQ) => ELSE LLAVE_IZQ          //4
            ({b==false}? (sentencia)* LLAVE_DER
            | {b==true}? (options{greedy=false;}:.)+ LLAVE_DER
            )
            )?
            ;
      exception
         catch [RecognitionException re] {
            mostrarExcepcion(re);
         }
   \end{lstlisting}

   \section{sentencia\_switch}
   La regla que a continuación se expone implementa las operaciones de bifurcación de casos, de tipo ``switch''. El valor que
   toma el acumulador se va incrementando a medida que se encuentran casos cuya expresión no se satisfacen.\\

   Si al llegar al token DEFAULT no ha ejecutado ningún caso, se reconocen las sentencias de la opción por defecto. De otra 
   manera las cláusulas encerradas entre llaves se obvian. 
   \begin{lstlisting}
      sentencia_switch 
      {Object resultado; int flag; int acumulador = 0; int contador = 0;} :
      
      SWITCH PAR_IZQ (resultado=expr_or) PAR_DER LLAVE_IZQ
      (flag = casos_switch[resultado] {acumulador += flag; contador++;})*  (DEFAULT DOBLE_PUNTO 
         {Processor.println(2, "Valor de acumulador:"+acumulador);} 
         LLAVE_IZQ
      ({ acumulador == contador}? (sentencia)* LLAVE_DER 
      //if flag is equal to contador has not run any case
         |{ acumulador < contador }? (options{greedy=false;}:.)+ LLAVE_DER )
      END_CASE FIN_INSTRUCCION)? LLAVE_DER ;
      exception
         catch [RecognitionException re] {
            mostrarExcepcion(re);
         }
   \end{lstlisting}

   \section{casos\_switch}
   Esta regla recibe como parámetro de entrada el resultado de la expresión condicional de la regla sentencia\_switch,
   y lo compara con otra expresión. Si no son equivalentes se obvian todas las cláusulas hasta llegar a la llave de cierre,
   y se devuelve 1 para indicar que el caso no se ha satisfecho.\\

   Esto permite a la regla padre ``sentencia\_switch'' saber si tiene que ejecutar las sentencias de la opción por defecto o no.
   \begin{lstlisting}
      casos_switch [Object resultado] returns [int n=0] 
      {Object res_eva; String cadena1 = null; String cadena2 = null;}: 
      CASE (res_eva = expr_or) DOBLE_PUNTO LLAVE_IZQ 
      {
      
      // ... 
      // Introduce en cadena1 el objeto resultado recibido como parámetro de 
      // entrada de la regla.
      // Introduce en cadena2 el resultado de la expresión res\_eva
      // ...  
      
      }
      
      ({(cadena2.toString()).compareTo(cadena1.toString()) == 0}? 
      (sentencia)* {n=0;} LLAVE_DER

            |{(cadena2.toString()).compareTo(cadena1.toString()) != 0}? 
      (options{greedy=false;}:.)+ {n=1;} LLAVE_DER )  
      END_CASE FIN_INSTRUCCION ;
   \end{lstlisting}

   \section{sentencia\_while}
   La regla mostrada a continuación implementa los bucles ``mientras''. Estos, pueden ser reconocidos de dos formas diferentes,
   atendiendo a que lleven paréntesis o no:
   \begin{enumerate}
      \item while de la forma: mientras (VERDAD) { sent = 1; sent = 2; var sentSent; }
      \item while de la forma: mientras (VERDAD) sent = 1;
   \end{enumerate}
   Una excepción es encargada de capturar los errores que se produzcan en el análisis sintáctico, consumiendo los tokens
   problemáticos hasta encontrar el siguiente fin de instrucción (método de pánico).
   \begin{lstlisting}
      sentencia_while
         {
            Boolean b = null; 
            Object expresion = null; 
            int marker = mark();
         } :
      
      (B_WHILE PAR_IZQ evaluar_expr PAR_DER LLAVE_IZQ) =>   //1
      B_WHILE PAR_IZQ expresion = evaluar_expr PAR_DER LLAVE_IZQ
      {
         b = ((Boolean)expresion).booleanValue();
            
      } 
      ({b==true}? (sentencia)* LLAVE_DER {rewind(marker);}
         | {b==false}? (options{greedy=false;}:.)+ LLAVE_DER) 
      |
      
      (B_WHILE PAR_IZQ evaluar_expr PAR_DER) =>        //2
      B_WHILE PAR_IZQ expresion = evaluar_expr PAR_DER
      {
         b = ((Boolean)expresion).booleanValue();
            
      } 
         ({b==true}? sentencia {rewind(marker);}
            | {b==false}? (options{greedy=false;}:.)+ FIN_INSTRUCCION )
      ;
      exception
         catch [RecognitionException re] {
            mostrarExcepcion(re);
         }
   \end{lstlisting}

   \section{sentencia\_for}
   La siguiente regla reconoce los bucles ``para'' de dos formas diferentes, según se utilicen corchetes para aglutinar varias
   sentencias o no:
   \begin{enumerate}
      \item Alternativa 1: para(identificador; expresion booleana; entero) { sent; sent2; }
      \item Alternativa 2: para(identificador; expresion booleana; entero) sent; 
   \end{enumerate}
  \textbf{NOTA}: El valor de identificador queda alterado después de la ejecución del bucle.\\
  Una excepción ANTLR es la encargada de capturar los errores sintácticos durante el análisis,  y gestionarlo.
   \begin{lstlisting}
      sentencia_for
         {
            Boolean b = null; 
            Boolean hecho = false;
            Object expresion = null; 
            int marker = mark();
            int numero = 0;
         } :
                           //1
      (B_FOR PAR_IZQ 
         IDENT FIN_INSTRUCCION 
         evaluar_expr FIN_INSTRUCCION 
         ENTERO PAR_DER LLAVE_IZQ) =>
      
         B_FOR PAR_IZQ id:IDENT FIN_INSTRUCCION 
         expresion = evaluar_expr FIN_INSTRUCCION 
         n:ENTERO PAR_DER LLAVE_IZQ
      {
         b = ((Boolean)expresion).booleanValue();
         
         if(hecho == false) {
         //Get value from variable id
         if(tablaSimbolos.existeSimbolo(id.getText())) {
            String contenido = 
               tablaSimbolos.getContenidoSimbolo(id.getText());
            
            if(contenido==null) {
               Processor.println(0,"Linea "+id.getLine()+
                  ": La variable no tiene asignado ningun valor "
                  +id.getText()); 
            }
            
            if(contenido.matches("[0-9~.]*")) {
               numero = new Integer(contenido.toString()).intValue();
               hecho = true;
            } 
         } else 
            Processor.println(0,"Linea "+id.getLine()+
               ": la variable no ha sido declarada "+id.getText());
         }   
         numero = numero + Integer.parseInt(n.getText());  
         // Saves the value
         tablaSimbolos.set(id, numero);
      } 
      ({b==true}? (sentencia)* LLAVE_DER {rewind(marker);}
         | {b==false}? (options{greedy=false;}:.)+ LLAVE_DER) 
         |
                        //2
      (B_FOR PAR_IZQ IDENT FIN_INSTRUCCION 
      evaluar_expr FIN_INSTRUCCION 
      ENTERO PAR_DER) => 
      
      B_FOR PAR_IZQ id2:IDENT FIN_INSTRUCCION 
      expresion = evaluar_expr FIN_INSTRUCCION 
      n2:ENTERO PAR_DER
      {
         b = ((Boolean)expresion).booleanValue();
         
         if(hecho == false) {
         // Get value from variable id2
         if(tablaSimbolos.existeSimbolo(id2.getText())) {
            String contenido =
               tablaSimbolos.getContenidoSimbolo(id2.getText());
               if(contenido==null) {
                  Processor.println(0,"Linea "+id2.getLine()+
                     ": La variable no tiene asignado ningun valor "+
                     id2.getText()); 
               }
            
               if(contenido.matches("[0-9~.]*")) {
                  numero = new Integer(contenido.toString()).intValue();
                  hecho = true;
               } 
            } else 
               Processor.println(0,"Linea "+id2.getLine()+
                  ": la variable no ha sido declarada "+id2.getText());
         }   
         numero = numero + Integer.parseInt(n2.getText());
            
         //Saves the value
         tablaSimbolos.set(id2, numero);
      } 
      ({b==true}? sentencia {rewind(marker);}
         |{b==false}? (options{greedy=false;}:.)+ FIN_INSTRUCCION) 
      ;
      exception
         catch [RecognitionException re] {
            mostrarExcepcion(re);
         }
   \end{lstlisting}

   \section{eliminar\_var}
   La siguiente regla es utilizada para la eliminación de variables declaradas. En caso de no existir tal variable se muestra
   un mensaje de error.
   \begin{lstlisting}
      eliminar_var {String res;}: SUP id:IDENT
      {
         res=tablaSimbolos.delSimbolo(id);
         if(res.compareTo(id.getText()) == 0)
            Processor.println(1, "Linea "+
               id.getLine()+": Variable \""+
               id.getText()+"\" ha sido eliminada");
         else 
            Processor.println(0, "Linea "+
               id.getLine()+": Variable \""+
               id.getText()+"\" no ha sido eliminada, no existe");
      };
   \end{lstlisting}

   \section{funcion\_sticky}
   La siguiente regla distingue entre aquellas que se actúan directamente sobre \textbf{\textit{Stickman}} o sobre el control del tiempo
   de la escena.
   \begin{lstlisting}
      funcion_sticky: f_tiempo | f_mover | f_flexionar | f_girar;
   \end{lstlisting}

   \section{f\_tiempo}
   La regla mostrada a continuación es la responsable de reconocer las instrucciones para el control del tiempo en la escena.\\
   \begin{itemize}
      \item Establecer el tiempo inicial de la escena.
      \item Avanzar algunos milisegundos en la escena.\\
   \end{itemize}
   Una excepción es la encargada de capturar los errores sintácticos, mostrar error y consumir los tokens hasta el final de instrucción.
   \begin{lstlisting}
      f_tiempo {Object res1;}: TIEMPO (est:ESTABLECE|AVANZA) PAR_IZQ  res1=expr_aritmetica PAR_DER
      {
      // Convert to correct format in order to call the function
      String cadena1 = res1.toString();
      float res1_float = Float.parseFloat(cadena1);
      int res1_int = Math.round(res1_float);

      Processor.println(1,"Entrando tiempo sticky. "+"res1_int: "+ res1_int);

      //If option is ESTABLECE, function "tiempo establece" is called.
      if(est != null) {
         Processor.println(1,"FSticky -> tiempo establece.");
         gui.StickMotion.scene.setTime(res1_int);
      }// Else "tiempo avanza" is called.
      else {
         Processor.println(1,"FSticky -> tiempo avanza.");
         gui.StickMotion.scene.addTime(res1_int);
         }
      };
      exception
         catch [RecognitionException re] {
            mostrarExcepcion(re);
         }
   \end{lstlisting}

   \section{f\_mover}
   La siguiente regla ha sido implementada para mover a Stickman por la escena, en coordenadas \textbf{\textit{x, y, z}}. La instrucción 
   precisa además un cuarto parámetro, el tiempo empleado en desarrollar el movimiento especificado. \\

   Una excepción es la encargada de capturar los errores sintácticos durante el análisis, mostrar error y consumir los tokens hasta el
   final de instrucción.
   \begin{lstlisting}
      f_mover {Object res1; Object res2; Object res3; Object res4;}: 
         mover:MOVER STICKMAN PAR_IZQ
         res1=expr_aritmetica SEPARA
         res2=expr_aritmetica SEPARA
         res3=expr_aritmetica SEPARA
         res4=expr_aritmetica PAR_DER 
         {
         Processor.println(1,"Entrando mover sticky.");
      
         // Convert to correct format in order to call the function
         // Exception in Java.
         try {
            String cadena1 = res1.toString(); 
            float res1_float = Float.parseFloat(cadena1);
            String cadena2 = res2.toString(); 
            float res2_float = Float.parseFloat(cadena2);
            String cadena3 = res3.toString(); 
            float res3_float = Float.parseFloat(cadena3);
            String cadena4 = res4.toString(); 
            float res4_float = Float.parseFloat(cadena4);
            int res4_int = Math.round(res4_float);
      
            gui.StickMotion.scene.moveStickman(res1_float, 
                        res2_float, 
                        res3_float, 
                        res4_int);
      }catch (NumberFormatException nfe) {
         Processor.println(0, "En linea " + mover.getLine() + ": 
         El parametro debe ser un numero." );
      } 
      catch (Exception e) {
         Processor.println(0, "En linea " + mover.getLine() + ": 
         " + e.toString() );
         }
      }; // Exception in ANTLR.
      exception
         catch [RecognitionException re] {
            mostrarExcepcion(re);
      }
   \end{lstlisting}

   \section{f\_flexionar}
   La regla especificada a continuación permite flexionar brazo o pierna, ya sea derecho/a o izquierdo/a. Los argumentos que
   necesita tal acción son dos expresiones aritméticas separadas entre comas y encerradas entre paréntesis, que se corresponden con:
   \begin{itemize}
      \item Sentido de giro, en radianes.
      \item Tiempo en milisegundos en el cual se prolongará el movimiento.
   \end{itemize}
   Una excepción ANTLR es la encargada de capturar los errores sintácticos durante el análisis, y gestionarlo.
   \begin{lstlisting}
      f_flexionar {Object res1; Object res2;}: 
         FLEXIONAR (BRAZO | PIERNA) (der:DER | IZQ) 
         PAR_IZQ res1=expr_aritmetica SEPARA res2=expr_aritmetica PAR_DER
      {
      Processor.println(1,"Entrando flexionar sticky.");
      
      // Convert to correct format in order to call the function
      String cadena1 = res1.toString(); 
      float res1_float = Float.parseFloat(cadena1);
      String cadena2 = res2.toString(); 
      float res2_float = Float.parseFloat(cadena2);
      int res2_int = Math.round(res2_float);

      //if BRAZO is selected, stiky's amr is moved, else PIERNA has 
      //been written and it will be moved.
      if(brazo != null) {
         //if DER is chosen, right arm is moved
         if (der != null) {
            Processor.println(1,"FSticky --> Brazo Derecho "+
            res1_float+"rad / "+res2_int+" secs.");
            gui.StickMotion.scene.flexRArm(res1_float, res2_int);
         }
         else {
            Processor.println(1,"FSticky --> Brazo Izquierdo "+
            res1_float+"rad / "+res2_int+" secs.");
            gui.StickMotion.scene.flexLArm(res1_float, res2_int);
         }
      }
      else {
         //if DER is chosen, right leg is moved
         if (der != null) {
            Processor.println(1,"FSticky --> Pierna Derecha "+
            res1_float+" / "+res2_int+" secs.");
            gui.StickMotion.scene.flexRLeg(res1_float, res2_int);
         }
         else {
            Processor.println(1,"FSticky --> Pierna Izquierda "+
            res1_float+" / "+res2_int+" secs.");
            gui.StickMotion.scene.flexLLeg(res1_float, res2_int);
         }
      }
      };
      exception
         catch [RecognitionException re] {
            mostrarExcepcion(re);
      }
   \end{lstlisting}

   \section{f\_girar}
   La siguiente regla permite el movimiento de rotación de las extremidades de Stickman, de todo el tronco o sólo de la cabeza.
   Se necesitan tres parámetros:
   \begin{itemize}
      \item Acimut (o sentido de giro sobre el plano horizontal).
      \item Sentido de giro en el plano vertical.
      \item Tiempo empleado en realizar la acción.
   \end{itemize}
   Una excepción ANTLR es la encargada de capturar los errores sintácticos durante el análisis,  y gestionarlo.
   \begin{lstlisting}
      f_girar {Object res1; Object res2; Object res3;}: 
         girar:GIRAR (stick:STICKMAN | cab:CABEZA | bra:BRAZO | PIERNA) 
               (der:DER | IZQ)?
         PAR_IZQ res1=expr_aritmetica 
         SEPARA res2=expr_aritmetica 
         SEPARA res3=expr_aritmetica 
         PAR_DER
         
      {
      try {
         // Convert to correct format in order to call the function
         String cadena1 = res1.toString(); 
         float res1_float = Float.parseFloat(cadena1);
         String cadena2 = res2.toString(); 
         float res2_float = Float.parseFloat(cadena2);
         String cadena3 = res3.toString(); 
         float res3_float = Float.parseFloat(cadena3);
         int res3_int = Math.round(res3_float);

         Processor.println(1,"Entrando girar sticky.");      

         if(stick != null) {
         Processor.println(1,"Gira Stickman 
         ("+res1_float+","+res2_float+")rad / "+res3_int+" secs.");
         gui.StickMotion.scene.rotateStickman( res1_float, 
                        res2_float, 
                        res3_int);
         } else if(cab != null) {
         Processor.println(1,"Gira Cabeza 
         ("+res1_float+","+res2_float+")rad / "+res3_int+" secs.");
         gui.StickMotion.scene.rotateHead(res1_float, res2_float, res3_int);
      } else if(bra != null)
         if( der != null ) {
            Processor.println(1,"Gira Brazo Derecho 
            ("+res1_float+","+res2_float+")rad / "+res3_int+" secs.");
            gui.StickMotion.scene.rotateRArm( res1_float, 
                        res2_float, 
                        res3_int);
         } else {
            Processor.println(1,"Gira Brazo Izquierdo 
            ("+res1_float+","+res2_float+")rad / "+res3_int+" secs.");
            gui.StickMotion.scene.rotateLArm( res1_float, 
                        res2_float, 
                        res3_int);
         } else 
            if( der != null ) {
            Processor.println(1,"Gira Pierna Derecha 
            ("+res1_float+","+res2_float+")rad / "+res3_int+"
               secs.");
            gui.StickMotion.scene.rotateRLeg( res1_float, 
                        res2_float,
                        res3_int);
            } else {
            Processor.println(1,"Gira Pierna Izquierda 
            ("+res1_float+","+res2_float+")rad / "+res3_int+"
            secs.");
            gui.StickMotion.scene.rotateLLeg( res1_float,
                        res2_float,
                        res3_int);
            }
      }catch (NumberFormatException nfe) {
         Processor.println(0, "En linea " + girar.getLine() + ": El 
               parametro debe ser un numero." );
      } 
      catch (Exception e) {
         Processor.println(0, "En linea " + girar.getLine() + ": " +
               e.toString() );
      }
      };
      exception
         catch [RecognitionException re] {
            mostrarExcepcion(re);
      }
   \end{lstlisting}

   \section{imprimir}
   La siguiente regla permite imprimir por pantalla cadenas y valores de identificadores, así como su concatenación. Una excepción
   ANTLR es la encargada de capturar los errores sintácticos durante el análisis, y gestionarlo.
   \begin{lstlisting}
      imprimir returns [String respuesta=null]
      { String expr1; String expr2;}:
      
      IMPRIMIR PAR_IZQ
      expr1 = impr_base { respuesta = expr1;}
      
      (OP_SUM expr2 = impr_base
      {
         respuesta = new String(expr1+expr2);
         expr1 = respuesta;
      })* PAR_DER
      ; // ANTLR Exception.
      exception
         catch [RecognitionException re] {
            mostrarExcepcion(re);
      }
   \end{lstlisting}

   \section{impr\_base}
   Esta regla define las expresiones base que son sujetas de ser impresas por la primitiva ``mostrar'', es decir, cadenas y valores
   almacenados en variables. Si la variable referenciada no existe en la tabla de símbolos se muestra un error en la ventana de 
   resultados de interpretación de la interfaz.
   \begin{lstlisting}
      impr_base returns [String respuesta=null]
      {Object e;}:
      (n1:CADENA 
      {
         // Elimina las comillas
      }
      |id:IDENT
      {
         // Rescata el valor de la variable si ha sido declarada. En caso 
         //contrario muestra mensaje de error...
      } )
      ;
      exception
         catch [RecognitionException re] {
            mostrarExcepcion(re);
         }
   \end{lstlisting}
