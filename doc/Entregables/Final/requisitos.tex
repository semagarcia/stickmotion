% Created 2010-06-02 Wed 13:04
\documentclass[11pt,a4paper]{report}
\usepackage{graphicx}
\usepackage[utf8]{inputenc}
\usepackage[spanish]{babel}
\usepackage{times}
\usepackage{amsmath}
\usepackage{amstext}
\usepackage{lscape}
\usepackage{color}
\setlength{\parindent}{7mm} %sangria la primera linea
\usepackage[pdftex=true,colorlinks=true,plainpages=false]{hyperref} % Soporte hipertexto

%PROFUNDIDAD DE LA TABLA DE CONTENIDO O ÍNDICE GENERAL
\setcounter{tocdepth}{4}
\setcounter{secnumdepth}{3}

\usepackage{graphicx}
\sloppy % suaviza las reglas de ruptura de líneas de LaTeX

\title{Stickmotion}
\author{grupo}
\date{Córdoba, 3 de Junio de 2010}

\begin{document}

\maketitle

\setcounter{tocdepth}{3}
\tableofcontents
\vspace*{1cm}






\chapter{Análisis de requisitos}
\label{sec-1}


\section{Introducción}
\label{sec-1.1}


Para clasificar y ordenar los requisitos es importante establecer una división de los distintos aspectos de la aplicación en diferentes bloques funcionales,
permitiendo una descripción ordenada de lo que se espera del programa desarrollado. \\

\begin{figure}[htb]
\centerline{\includegraphics[width=\textwidth]{descFuncional}}
\caption{\label{fig:dfuncional}Esquema de la descomposición en distintos bloques funcionales.}
\end{figure}


En la figura \ref{fig:dfuncional} se muestra un esquema de la descomposición funcional que se llevará a cabo. Los distintos bloques se describirán a
continuación. \\

\subsection{Motor 3D de animaciones}
\label{sec-1.1.1}


Este bloque será el encargado de llevar a cabo la creación y gestión de la escena 3D, siendo el responsable del renderizado del Stickman que se deberá
mostrar en la interfaz, así como realizar y almacenar las animaciones que se deberán realizar definidas por el lenguaje. \\

Deberá por tanto ofrecer una interfaz con funciones necesarias para la inclusión de animaciones en el Stickman, y obtener como salida una escena 3D que
pueda ser mostrada en la interfaz gráfica. \\

\subsection{Procesador del lenguaje Sticky}
\label{sec-1.1.2}


Este bloque funcional será el cerebro de la interpretación del lenguaje Sticky diseñado, implementará las gramáticas empleando la librería ANTLR, y lo hará
en base al lenguaje que se ha diseñado para tal efecto, al cual se le ha denominado ``Sticky''.\\

Deberá por tanto recibir el código introducido por el usuario y emplearlo procesando las acciones pertinentes para comunicarse con el motor 3D de
animaciones en la realización de los movimientos que se especifiquen en el código Sticky. \\

\section{Requisitos funcionales}
\label{sec-1.2}


A continuación se enumerarán una serie de requisitos especificando las funciones que deberán realizar cada uno de los componentes en los que se ha dividido
el diseño de la aplicación. \\

\subsection{Motor 3D de animaciones}
\label{sec-1.2.1}


\begin{itemize}
\item \textbf{RF1:} Deberá ofrecer la creación de una escena 3D que contenga la figura del Stickman.
\item \textbf{RF2:} La figura Stickman deberá poseer 4 extremidades y 1 cabeza que podrán rotarse en cualquier dirección del espacio tridimensional.
  \begin{itemize}
  \item \textbf{RF2.1:} La rotación se realizará indicando el azimut y la elevación, en radianes, que se desea llevar a cabo, y el tiempo, en milisegundos,
			que durará la misma.
  \end{itemize}
\item \textbf{RF3:} El sistema debe permitir la rotación de la figura Stickman.
  \begin{itemize}
  \item \textbf{RF3.1:} La rotación se realizará indicando el azimut y la elevación, en radianes, que se desea llevar a cabo, y el tiempo, en milisegundos,
			que durará la misma.
  \end{itemize}
\item \textbf{RF4:} El sistema debe permitir la traslación de la figura Stickman.
  \begin{itemize}
  \item \textbf{RF4.1:} La traslación se realizará indicando el desplazamiento en X, Y y Z, siendo la unidad la medida del antebrazo del Stickman, y el
			tiempo en milisegundos que durará la misma.
  \end{itemize}
\item \textbf{RF5:} Cada una de las 4 extremidades del Stickman poseerá un punto de flexión sobre el cual podrá doblarse.
  \begin{itemize}
  \item \textbf{RF5.1:} La flexión se realizará indicando un ángulo en radianes y el tiempo en segundos que durará la misma.
  \end{itemize}
\item \textbf{RF6:} La animación debe ser temporal, por lo que se debe almacenar el tiempo de animación actual.
  \begin{itemize}
  \item \textbf{RF6.1:} El sistema debe facilitar el tiempo actual de animación cuando el usuario lo desee.
  \item \textbf{RF6.2:} El sistema debe permitir la modificación del tiempo actual de animación cuando el usuario lo desee.
  \end{itemize}
\end{itemize}
\subsection{Procesador del lenguaje Sticky}
\label{sec-1.2.2}


\begin{itemize}
\item \textbf{RF6:} Se seguirán las construcciones gramaticales, control de flujo y operaciones especificadas por la descripción del lenguaje Sticky.
\item \textbf{RF7:} Se ofrecerá una salida amigable en cuanto al reporte de errores, permitiendo en lo posible, la recuperación tras los mismos para
		    continuar evaluando el resto del código.
\item \textbf{RF8:} Para facilitar la depuración, se ofrecerá un nivel regulable de detalle de la salida esperada, de forma que se muestren distintos
		    mensajes de depuración o ``DEBUG'' si el usuario desea un mayor detalle sobre las operaciones que el procesador está realizando.
\end{itemize}
\subsection{Interfaz gráfica y editor}
\label{sec-1.2.3}


\begin{itemize}
\item \textbf{RF9:} El sistema debe permitir la creación de nuevos documentos con código Sticky.
\item \textbf{RF10:} El sistema debe permitir la carga de archivos de código Sticky (extensión .stk) en la interfaz del editor.
\item \textbf{RF11:} El sistema debe permitir almacenar el código Sticky insertado en el editor en un archivo en disco.
\item \textbf{RF12:} El sistema debe permitir deshacer cambios sobre el código en el editor.
\item \textbf{RF13:} El sistema debe permitir rehacer cambios sobre el código en el editor.
\item \textbf{RF14:} El sistema debe permitir copiar, cortar y pegar porciones de código en el editor.
\item \textbf{RF15:} El sistema debe permitir iniciar la interpretación del código contenido en el editor.
\item \textbf{RF16:} El sistema debe permitir detener una interpretación en curso.
\item \textbf{RF17:} El sistema debe permitir limpiar el editor de código.
\item \textbf{RF18:} El sistema debe generar código cuando el usuario lo desee. El código que debe poder generar son el condicional ``si`` y los bucles
		    ''mientras`` y ''para``.
\item \textbf{RF19:} El sistema debe permitir editar el código Sticky por teclado desde la propia interfaz del editor.
\item \textbf{RF20:} La interfaz debe mostrar mostrar la escena 3D del Stickman en animación.
\end{itemize}
\section{Requisitos no funcionales}
\label{sec-1.3}


Para poder llevar a cabo todos los objetivos correctamente nuestra aplicación deberá cumplir con la siguiente serie de requisitos no funcionales: \\


\begin{itemize}
\item \textbf{RNF1:} El programa deberá ser desarrollado en Java, siguiendo el programa de la asignatura, y ofreciendo además portabilidad en cuanto a
		      plataforma de ejecución.
\item \textbf{RNF2:} El procesamiento del lenguaje deberá realizarse empleando la librería ANTLR para la creación de reconocedores de lenguaje.
\item \textbf{RNF3:} La aplicación deberá ofrecer un manejo intuitivo y sencillo.
\end{itemize}

\end{document}