\documentclass[twoside,a4paper,11pt]{book}
\usepackage{anysize}
\marginsize{3cm}{2cm}{2.5cm}{2.5cm}
\begin{document}


\subsection{Especificación de casos de uso}
A continuación se especifican los actores que harán uso del sistema con el fin de identificar todos los casos de uso a desarrollar y posteriormente se
expone una descripción tabulada de los casos de uso y diagramas para su mejor comprensión.\\
La notación tabular elegida es la siguiente:\\


%% Tabla Especificación de casos de uso.
\begin{table}[!ht]
    \centering
    \begin{tabular}{|p{4cm}|p{11.5cm}|}
    \hline

    \textbf{Caso de uso} & \textit{Nombre del caso de uso.}\\
    \hline

    \textbf{Objetivo} & \textit{Objetivo principal del caso de uso.}\\
    \hline

    \textbf{Actores} & \textit{Actores implicados en el caso de uso.}\\
    \hline

    \textbf{Disparador} & \textit{Descripción de la acción que provoca la ejecución del caso de uso.}\\
    \hline

    \textbf{Precondiciones} & \textit{Condiciones que se deben de dar para la ejecución del caso de uso.}\\
    \hline

    \textbf{Descripción} & \textit{Breve descripción del comportamiento del caso de uso.}\\
    \hline

    \multicolumn{2}{|c|}{\textbf{Curso normal de los eventos}}\\
    \hline

    \end{tabular}
    \begin{tabular}{|p{7.75cm}|p{7.75cm}|}
    \hspace{2cm}\textbf{Acción de los actores} & \hspace{1.75cm}\textbf{Respuesta del sistema}\\
    \hline

    \textit{Acciones que realizan los actores del caso de uso.} & \textit{Acciones que realiza el sistema.}\\
    \hline

    \end{tabular}

    \begin{tabular}{|p{15.9cm}|}
      \hspace{6cm}\textbf{Cursos alternativos}\\
      \hline

    \textit{Acciones a realizar fuera del curso normal de eventos.}\\
    \hline
    \end{tabular}
    \caption{Especificación de casos de uso.}
\end{table}


\subsubsection{Identificación de actores}
Nuestro sistema sólo contará con un actor, que será el usuario del mismo y podrá realizar todas las operaciones disponibles.\\

IMAGEN actores.jpg
%%\caption{Identificación de actores.}


\subsubsection{Casos de uso}
El siguiente diagrama muestra los casos de uso identificados que se desarrollan.\\

IMAGEN dcu.jpg
%%\caption{Diagrama de casos de uso.}

%% Tabla Caso de uso 1 - Editar código.
\begin{table}[!ht]
    \centering
    \begin{tabular}{|p{4cm}|p{11.5cm}|}
    \hline

    \textbf{Caso de uso} & \textit{Editar código.}\\
    \hline

    \textbf{Objetivo} & \textit{Editar el código a interpretar.}\\
    \hline

    \textbf{Actores} & \textit{Usuario.}\\
    \hline

    \textbf{Disparador} & \textit{Este caso de uso es disparado cada vez que el usuario realiza alguna modificación sobre el código a interpretar.}\\
    \hline

    \textbf{Precondiciones} & \textit{Ninguna.}\\
    \hline

    \textbf{Descripción} & \textit{El usuario edita el código a interpretar.}\\
    \hline

    \multicolumn{2}{|c|}{\textbf{Curso normal de los eventos}}\\
    \hline

    \end{tabular}
    \begin{tabular}{|p{7.75cm}|p{7.75cm}|}
    \hspace{2cm}\textbf{Acción de los actores} & \hspace{1.75cm}\textbf{Respuesta del sistema}\\
    \hline

    & 
    \textit{1.- El sistema da la posibilidad de editar el código mediante:}
   
      \begin{itemize}
	\item \textit{Abrir fichero.}
	\item \textit{Guardar fichero.}
	\item \textit{Escribir.}
	\item \textit{Nuevo documento.}
      \end{itemize}
    \\
    \hline

    \end{tabular}

    \begin{tabular}{|p{15.9cm}|}
      \hspace{6cm}\textbf{Cursos alternativos}\\
      \hline

    \textit{Ninguno.}\\
    \hline

    \end{tabular}
    \caption{Caso de uso 1.- Editar código.}
\end{table}

IMAGEN cu1.jpg
%%\caption{Caso de uso 1.- Editar código.}

%% Tabla Caso de uso 1.1 - Abrir fichero.
\begin{table}[!ht]
    \centering
    \begin{tabular}{|p{4cm}|p{11.5cm}|}
    \hline

    \textbf{Caso de uso} & \textit{Abrir fichero.}\\
    \hline

    \textbf{Objetivo} & \textit{Abrir un fichero de texto con código Sticky.}\\
    \hline

    \textbf{Actores} & \textit{Usuario.}\\
    \hline

    \textbf{Disparador} & \textit{Este caso de uso se dispara cuando el usuario selecciona la opción correspondiente para abrir un fichero.}\\
    \hline

    \textbf{Precondiciones} & \textit{Ninguna.}\\
    \hline

    \textbf{Descripción} & \textit{El sistema abre un archivo de texto con código Sticky.}\\
    \hline

    \multicolumn{2}{|c|}{\textbf{Curso normal de los eventos}}\\
    \hline

    \end{tabular}
    \begin{tabular}{|p{7.75cm}|p{7.75cm}|}
    \hspace{2cm}\textbf{Acción de los actores} & \hspace{1.75cm}\textbf{Respuesta del sistema}\\
    \hline

    \textit{1.- El usuario selecciona un fichero para abrir de sus sistema de ficheros.}
    &
    \textit{2.- El sistema comprueba el fichero.}

    \textit{3.- El sistema carga el fichero.}

    \textit{4.- El sistema limpia el editor y muestra la información contenida en el fichero cargado.}

    \\
    \hline
    \end{tabular}

    \begin{tabular}{|p{15.9cm}|}
      \hspace{6cm}\textbf{Cursos alternativos}\\
      \hline
     
	\textit{2a.- El fichero no tiene extensión .stk}

	\textit{  2a1.- El sistema finaliza este caso de uso.}
      \\
      \hline
    \end{tabular}
    \caption{Caso de uso 1.1.- Abrir fichero.}
\end{table}


%% Tabla Caso de uso 1.2 - Guardar fichero.
\begin{table}[!ht]
    \centering
    \begin{tabular}{|p{4cm}|p{11.5cm}|}
    \hline

    \textbf{Caso de uso} & \textit{Guardar fichero.}\\
    \hline

    \textbf{Objetivo} & \textit{Guarda un fichero con código Sticky.}\\
    \hline

    \textbf{Actores} & \textit{Usuario.}\\
    \hline

    \textbf{Disparador} & \textit{Este caso de uso se dispara cuando el usuario selecciona la opción correspondiente para guardar un fichero.}\\
    \hline

    \textbf{Precondiciones} & \textit{Ninguna.}\\
    \hline

    \textbf{Descripción} & \textit{El sistema guarda un archivo de texto con código Sticky.}\\
    \hline

    \multicolumn{2}{|c|}{\textbf{Curso normal de los eventos}}\\
    \hline

    \end{tabular}
    \begin{tabular}{|p{7.75cm}|p{7.75cm}|}
    \hspace{2cm}\textbf{Acción de los actores} & \hspace{1.75cm}\textbf{Respuesta del sistema}\\
    \hline

    \textit{1.- El usuario selecciona la ruta y el nombre del fichero.}
    &
    \textit{2.- El sistema almacena el fichero.}
    \\
    \hline
    \end{tabular}

    \begin{tabular}{|p{15.9cm}|}
      \hspace{6cm}\textbf{Cursos alternativos}\\
      \hline
     
	\textit{2a.- El fichero no puede ser almacenado.}

	\textit{  2a1.- El sistema finaliza este caso de uso.}
      \\
      \hline
    \end{tabular}
    \caption{Caso de uso 1.2.- Guardar fichero.}
\end{table}


%% Tabla Caso de uso 1.3 - Escribir.
\begin{table}[!ht]
    \centering
    \begin{tabular}{|p{4cm}|p{11.5cm}|}
    \hline

    \textbf{Caso de uso} & \textit{Escribir.}\\
    \hline

    \textbf{Objetivo} & \textit{Escribir código Sticky en el editor.}\\
    \hline

    \textbf{Actores} & \textit{Usuario.}\\
    \hline

    \textbf{Disparador} & \textit{Este caso de uso se dispara cuando el usuario realiza alguna operación sobre el editor.}\\
    \hline

    \textbf{Precondiciones} & \textit{Ninguna.}\\
    \hline

    \textbf{Descripción} & \textit{Modifica el contenido del editor de código.}\\
    \hline

    \multicolumn{2}{|c|}{\textbf{Curso normal de los eventos}}\\
    \hline

    \end{tabular}
    \begin{tabular}{|p{7.75cm}|p{7.75cm}|}
    \hspace{2cm}\textbf{Acción de los actores} & \hspace{1.75cm}\textbf{Respuesta del sistema}\\
    \hline

    \textit{1.- El usuario modifica el código en el editor.}
    &
    \\
    \hline
    \end{tabular}

    \begin{tabular}{|p{15.9cm}|}
      \hspace{6cm}\textbf{Cursos alternativos}\\
      \hline     
	\textit{Ninguno.}
      \\
      \hline
    \end{tabular}
    \caption{Caso de uso 1.3.- Escribir.}
\end{table}


%% Tabla Caso de uso 1.4 - Deshacer.
\begin{table}[!ht]
    \centering
    \begin{tabular}{|p{4cm}|p{11.5cm}|}
    \hline

    \textbf{Caso de uso} & \textit{Deshacer.}\\
    \hline

    \textbf{Objetivo} & \textit{Deshacer el útlimo cambio realizado sobre el código en el editor.}\\
    \hline

    \textbf{Actores} & \textit{Usuario.}\\
    \hline

    \textbf{Disparador} & \textit{Este caso de uso se dispara cuando el usuario desea deshacer un cambio en el editor.}\\
    \hline

    \textbf{Precondiciones} & \textit{Debe haberse realizado algún cambio sobre el código.}\\
    \hline

    \textbf{Descripción} & \textit{Deshace el último cambio realizado en el editor de código.}\\
    \hline

    \multicolumn{2}{|c|}{\textbf{Curso normal de los eventos}}\\
    \hline

    \end{tabular}
    \begin{tabular}{|p{7.75cm}|p{7.75cm}|}
    \hspace{2cm}\textbf{Acción de los actores} & \hspace{1.75cm}\textbf{Respuesta del sistema}\\
    \hline

    &
    \textit{1.- El sistema deshace el último cambio realizado sobre el editor.}
    \\
    \hline
    \end{tabular}

    \begin{tabular}{|p{15.9cm}|}
      \hspace{6cm}\textbf{Cursos alternativos}\\
      \hline     
	\textit{Ninguno.}
      \\
      \hline
    \end{tabular}
    \caption{Caso de uso 1.4.- Deshacer.}
\end{table}


%% Tabla Caso de uso 1.5 - Rehacer.
\begin{table}[!ht]
    \centering
    \begin{tabular}{|p{4cm}|p{11.5cm}|}
    \hline

    \textbf{Caso de uso} & \textit{Rehacer.}\\
    \hline

    \textbf{Objetivo} & \textit{Rehacer el útlimo cambio deshecho.}\\
    \hline

    \textbf{Actores} & \textit{Usuario.}\\
    \hline

    \textbf{Disparador} & \textit{Este caso de uso se dispara cuando el usuario desea rehacer un cambio deshecho anteriormente.}\\
    \hline

    \textbf{Precondiciones} & \textit{Debe haberse ejecutado anteriormente el caso de uso 1.4.- Deshacer.}\\
    \hline

    \textbf{Descripción} & \textit{Rehace el último cambio deshecho.}\\
    \hline

    \multicolumn{2}{|c|}{\textbf{Curso normal de los eventos}}\\
    \hline

    \end{tabular}
    \begin{tabular}{|p{7.75cm}|p{7.75cm}|}
    \hspace{2cm}\textbf{Acción de los actores} & \hspace{1.75cm}\textbf{Respuesta del sistema}\\
    \hline

    &
    \textit{1.- El sistema rehace el último cambio deshecho.}
    \\
    \hline
    \end{tabular}

    \begin{tabular}{|p{15.9cm}|}
      \hspace{6cm}\textbf{Cursos alternativos}\\
      \hline     
	\textit{Ninguno.}
      \\
      \hline
    \end{tabular}
    \caption{Caso de uso 1.5.- Rehacer.}
\end{table}


%% Tabla Caso de uso 1.6 - Cortar.
\begin{table}[!ht]
    \centering
    \begin{tabular}{|p{4cm}|p{11.5cm}|}
    \hline

    \textbf{Caso de uso} & \textit{Cortar.}\\
    \hline

    \textbf{Objetivo} & \textit{Cortar una porción de código y almacenarla en el portapepeles para su posterior inserción.}\\
    \hline

    \textbf{Actores} & \textit{Usuario.}\\
    \hline

    \textbf{Disparador} & \textit{Este caso de uso se dispara cuando el usuario desea cortar una porción de código.}\\
    \hline

    \textbf{Precondiciones} & \textit{Ninguna.}\\
    \hline

    \textbf{Descripción} & \textit{Corta un porción de código y la almacena en el portapapeles.}\\
    \hline

    \multicolumn{2}{|c|}{\textbf{Curso normal de los eventos}}\\
    \hline

    \end{tabular}
    \begin{tabular}{|p{7.75cm}|p{7.75cm}|}
    \hspace{2cm}\textbf{Acción de los actores} & \hspace{1.75cm}\textbf{Respuesta del sistema}\\
    \hline

    \textit{1.- El usuario selecciona una porción de código.}

    \textit{2.- El usuario corta el código seleccionado.}
    &
    \textit{3.- El sistema elimina la porción de código seleccionada por el usuario del editor.}

    \textit{4.- El sistema almacena la porción de código para su posterior inserción.}
    \\
    \hline
    \end{tabular}

    \begin{tabular}{|p{15.9cm}|}
      \hspace{6cm}\textbf{Cursos alternativos}\\
      \hline     
	\textit{Ninguno.}
      \\
      \hline
    \end{tabular}
    \caption{Caso de uso 1.6.- Cortar.}
\end{table}


%% Tabla Caso de uso 1.7 - Copiar.
\begin{table}[!ht]
    \centering
    \begin{tabular}{|p{4cm}|p{11.5cm}|}
    \hline

    \textbf{Caso de uso} & \textit{Copiar.}\\
    \hline

    \textbf{Objetivo} & \textit{Almacenar una porción de código en el portapepeles para su posterior inserción.}\\
    \hline

    \textbf{Actores} & \textit{Usuario.}\\
    \hline

    \textbf{Disparador} & \textit{Este caso de uso se dispara cuando el usuario desea copiar una porción de código.}\\
    \hline

    \textbf{Precondiciones} & \textit{Ninguna.}\\
    \hline

    \textbf{Descripción} & \textit{Almacena en el portapapeles una porción de código.}\\
    \hline

    \multicolumn{2}{|c|}{\textbf{Curso normal de los eventos}}\\
    \hline

    \end{tabular}
    \begin{tabular}{|p{7.75cm}|p{7.75cm}|}
    \hspace{2cm}\textbf{Acción de los actores} & \hspace{1.75cm}\textbf{Respuesta del sistema}\\
    \hline

    \textit{1.- El usuario selecciona una porción de código.}

    \textit{2.- El usuario copia el código seleccionado.}
    &
    \textit{3.- El sistema almacena la porción de código para su posterior inserción.}
    \\
    \hline
    \end{tabular}

    \begin{tabular}{|p{15.9cm}|}
      \hspace{6cm}\textbf{Cursos alternativos}\\
      \hline     
	\textit{Ninguno.}
      \\
      \hline
    \end{tabular}
    \caption{Caso de uso 1.7.- Copiar.}
\end{table}


%% Tabla Caso de uso 1.8 - Pegar.
\begin{table}[!ht]
    \centering
    \begin{tabular}{|p{4cm}|p{11.5cm}|}
    \hline

    \textbf{Caso de uso} & \textit{Pegar.}\\
    \hline

    \textbf{Objetivo} & \textit{Insertar la porción de código almacenada en el portapapeles en el editor.}\\
    \hline

    \textbf{Actores} & \textit{Usuario.}\\
    \hline

    \textbf{Disparador} & \textit{Este caso de uso se dispara cuando el usuario desea pegar una porción de código.}\\
    \hline

    \textbf{Precondiciones} & \textit{Debe haber alguna cadena almacenada en el portapapeles.}\\
    \hline

    \textbf{Descripción} & \textit{Inserta en el editor una porción de código.}\\
    \hline

    \multicolumn{2}{|c|}{\textbf{Curso normal de los eventos}}\\
    \hline

    \end{tabular}
    \begin{tabular}{|p{7.75cm}|p{7.75cm}|}
    \hspace{2cm}\textbf{Acción de los actores} & \hspace{1.75cm}\textbf{Respuesta del sistema}\\
    \hline

    \textit{1.- El usuario sitúa el cursor en alguna posición del editor.}

    \textit{2.- El usuario pega el contenido del portapapeles.}
    &
    \textit{3.- El sistema inserta el contenido del portapapeles en la posición indicada por el cursor del editor.}
    \\
    \hline
    \end{tabular}

    \begin{tabular}{|p{15.9cm}|}
      \hspace{6cm}\textbf{Cursos alternativos}\\
      \hline     
	\textit{Ninguno.}
      \\
      \hline
    \end{tabular}
    \caption{Caso de uso 1.8.- Pegar.}
\end{table}


%% Tabla Caso de uso 1.9 - Insertar código.
\begin{table}[!ht]
    \centering
    \begin{tabular}{|p{4cm}|p{11.5cm}|}
    \hline

    \textbf{Caso de uso} & \textit{Insertar código.}\\
    \hline

    \textbf{Objetivo} & \textit{Insertar código en el editor.}\\
    \hline

    \textbf{Actores} & \textit{Usuario.}\\
    \hline

    \textbf{Disparador} & \textit{Este caso de uso se dispara cuando el usuario escribe sobre el editor o selecciona la opción correspondiente para
				  generar código desde la interfaz.}\\
    \hline

    \textbf{Precondiciones} & \textit{Ninguna.}\\
    \hline

    \textbf{Descripción} & \textit{Escribe código en el editor.}\\
    \hline

    \multicolumn{2}{|c|}{\textbf{Curso normal de los eventos}}\\
    \hline

    \end{tabular}
    \begin{tabular}{|p{7.75cm}|p{7.75cm}|}
    \hspace{2cm}\textbf{Acción de los actores} & \hspace{1.75cm}\textbf{Respuesta del sistema}\\
    \hline

    \textit{1.- El usuario selecciona la forma de introducir código en el editor:}

      \textit{	Escribiendo sobre el mismo.}

      \textit{	Generando código(Bucles ``mientras``, ``para`` o condición ``si``) desde la interfaz.}
    &
    \textit{2.- El sistema inserta el código correspondiente en el editor.}
    \\
    \hline
    \end{tabular}

    \begin{tabular}{|p{15.9cm}|}
      \hspace{6cm}\textbf{Cursos alternativos}\\
      \hline     
	\textit{Ninguno.}
      \\
      \hline
    \end{tabular}
    \caption{Caso de uso 1.9.- Insertar código.}
\end{table}


%% Tabla Caso de uso 1.10 - Limpiar.
\begin{table}[!ht]
    \centering
    \begin{tabular}{|p{4cm}|p{11.5cm}|}
    \hline

    \textbf{Caso de uso} & \textit{Limpiar.}\\
    \hline

    \textbf{Objetivo} & \textit{Limpiar el contenido del editor de código.}\\
    \hline

    \textbf{Actores} & \textit{Usuario.}\\
    \hline

    \textbf{Disparador} & \textit{Este caso de uso se dispara cuando el usuario selecciona la opción correspondiente para limpiar el código.}\\
    \hline

    \textbf{Precondiciones} & \textit{Ninguna.}\\
    \hline

    \textbf{Descripción} & \textit{Limpia el contenido del editor de código.}\\
    \hline

    \multicolumn{2}{|c|}{\textbf{Curso normal de los eventos}}\\
    \hline

    \end{tabular}
    \begin{tabular}{|p{7.75cm}|p{7.75cm}|}
    \hspace{2cm}\textbf{Acción de los actores} & \hspace{1.75cm}\textbf{Respuesta del sistema}\\
    \hline

    &
    \textit{1.- El sistema limpiar el contenido del editor de código.}
    \\
    \hline
    \end{tabular}

    \begin{tabular}{|p{15.9cm}|}
      \hspace{6cm}\textbf{Cursos alternativos}\\
      \hline     
	\textit{Ninguno.}
      \\
      \hline
    \end{tabular}
    \caption{Caso de uso 1.10.- Limpiar.}
\end{table}


%% Tabla Caso de uso 1.11 - Nuevo documento.
\begin{table}[!ht]
    \centering
    \begin{tabular}{|p{4cm}|p{11.5cm}|}
    \hline

    \textbf{Caso de uso} & \textit{Nuevo documento.}\\
    \hline

    \textbf{Objetivo} & \textit{Crear un nuevo documento.}\\
    \hline

    \textbf{Actores} & \textit{Usuario.}\\
    \hline

    \textbf{Disparador} & \textit{Este caso de uso se dispara cuando el usuario selecciona la opción correspondiente para realizar un nuevo documento.}\\
    \hline

    \textbf{Precondiciones} & \textit{Ninguna.}\\
    \hline

    \textbf{Descripción} & \textit{Crear un nuevo documento.}\\
    \hline

    \multicolumn{2}{|c|}{\textbf{Curso normal de los eventos}}\\
    \hline

    \end{tabular}
    \begin{tabular}{|p{7.75cm}|p{7.75cm}|}
    \hspace{2cm}\textbf{Acción de los actores} & \hspace{1.75cm}\textbf{Respuesta del sistema}\\
    \hline

    \textit{2.- El usuario ejecuta el caso de uso 1.2.- Guardar fichero si desea almacenar el fichero.}
    &
    \textit{1.- El sistema da la posibilidad al usuario de almacenar en disco el documento actual si no lo está ya.}

    \textit{3.- Se ejecuta el caso de uso 1.10.- Limpiar.}
    \\
    \hline
    \end{tabular}

    \begin{tabular}{|p{15.9cm}|}
      \hspace{6cm}\textbf{Cursos alternativos}\\
      \hline     
	\textit{1a.- El documento actual ya está almacenado.}

	\textit{  1a1.- El sistema salta al paso 3.}

	\textit{2a.- El usuario cancela la operación.}

	\textit{  2a1.- Finaliza éste caso de uso y el sistema vuelve al estado en el que se encontraba antes de ejecutarlo.}

	\textit{2b.- El usuario no desea almacenar el documento anterior.}

	\textit{  2b1.- El sistema salta al paso 3.}
      \\
      \hline
    \end{tabular}
    \caption{Caso de uso 1.11.- Nuevo documento.}
\end{table}


%% Tabla Caso de uso 2 - Interpretar código.
\begin{table}[!ht]
    \centering
    \begin{tabular}{|p{4cm}|p{11.5cm}|}
    \hline

    \textbf{Caso de uso} & \textit{Interpretar código.}\\
    \hline

    \textbf{Objetivo} & \textit{Interpretar el código contenido en el editor.}\\
    \hline

    \textbf{Actores} & \textit{Usuario.}\\
    \hline

    \textbf{Disparador} & \textit{Este caso de uso es disparado cada vez que el usuario selecciona alguna opción de interpretación de código.}\\
    \hline

    \textbf{Precondiciones} & \textit{Ninguna.}\\
    \hline

    \textbf{Descripción} & \textit{Establece las opciones de depuración e interpreta el código.}\\
    \hline

    \multicolumn{2}{|c|}{\textbf{Curso normal de los eventos}}\\
    \hline

    \end{tabular}
    \begin{tabular}{|p{7.75cm}|p{7.75cm}|}
    \hspace{2cm}\textbf{Acción de los actores} & \hspace{1.75cm}\textbf{Respuesta del sistema}\\
    \hline

    & 
    \textit{1.- El sistema da la posibilidad de:}
   
      \begin{itemize}
	\item \textit{Iniciar interpretación.}
	\item \textit{Seleccionar nivel de depuración.}
      \end{itemize}
    \\
    \hline

    \end{tabular}

    \begin{tabular}{|p{15.9cm}|}
      \hspace{6cm}\textbf{Cursos alternativos}\\
      \hline

    \textit{Ninguno.}\\
    \hline

    \end{tabular}
    \caption{Caso de uso 2.- Interpretar código.}
\end{table}

IMAGEN cu2.jpg
%%\caption{Caso de uso 2.- Interpretar código.}


%% Tabla Caso de uso 2.1 - Iniciar intepretación.
\begin{table}[!ht]
    \centering
    \begin{tabular}{|p{4cm}|p{11.5cm}|}
    \hline

    \textbf{Caso de uso} & \textit{Iniciar interpretación.}\\
    \hline

    \textbf{Objetivo} & \textit{Interpretar el código contenido en el editor y mostrar los resultados de la interpretación.}\\
    \hline

    \textbf{Actores} & \textit{Usuario.}\\
    \hline

    \textbf{Disparador} & \textit{Este caso de uso se dispara cuando el usuario selecciona la opción de interpretar código.}\\
    \hline

    \textbf{Precondiciones} & \textit{Ninguna.}\\
    \hline

    \textbf{Descripción} & \textit{Interpreta el código contenido en el editor y muestra los resultados.}\\
    \hline

    \multicolumn{2}{|c|}{\textbf{Curso normal de los eventos}}\\
    \hline

    \end{tabular}
    \begin{tabular}{|p{7.75cm}|p{7.75cm}|}
    \hspace{2cm}\textbf{Acción de los actores} & \hspace{1.75cm}\textbf{Respuesta del sistema}\\
    \hline

    &
    \textit{1.- El sistema interpreta el código.}

    \textit{2.- El sistema carga la animación del Stickman correspondiente a la interpretación.}

    \textit{3.- El sistema muestra los mensajes correspondientes a la interpretación del código.}
    \\
    \hline
    \end{tabular}

    \begin{tabular}{|p{15.9cm}|}
      \hspace{6cm}\textbf{Cursos alternativos}\\
      \hline     
	\textit{Ninguno.}
      \\
      \hline
    \end{tabular}
    \caption{Caso de uso 2.1.- Iniciar interpretación.}
\end{table}


%% Tabla Caso de uso 2.2 - Detener intepretación.
\begin{table}[!ht]
    \centering
    \begin{tabular}{|p{4cm}|p{11.5cm}|}
    \hline

    \textbf{Caso de uso} & \textit{Detener interpretación.}\\
    \hline

    \textbf{Objetivo} & \textit{Detener la interpretación de código.}\\
    \hline

    \textbf{Actores} & \textit{Usuario.}\\
    \hline

    \textbf{Disparador} & \textit{Este caso de uso se dispara cuando el usuario selecciona la opción de detener la interpretación de código que se está
				  llevando a cabo.}\\
    \hline

    \textbf{Precondiciones} & \textit{Para que esta caso de uso se puede llevar a cabo, el caso de uso 2.1.- Interpretar código tiene que estar
				      ejecutándose.}\\
    \hline

    \textbf{Descripción} & \textit{Detiene la interpretación de código que se está llevando a cabo.}\\
    \hline

    \multicolumn{2}{|c|}{\textbf{Curso normal de los eventos}}\\
    \hline

    \end{tabular}
    \begin{tabular}{|p{7.75cm}|p{7.75cm}|}
    \hspace{2cm}\textbf{Acción de los actores} & \hspace{1.75cm}\textbf{Respuesta del sistema}\\
    \hline

    &
    \textit{1.- El sistema detiene la interpretación de código que se esté llevando a cabo.}
    \\
    \hline
    \end{tabular}

    \begin{tabular}{|p{15.9cm}|}
      \hspace{6cm}\textbf{Cursos alternativos}\\
      \hline     
	\textit{Ninguno.}
      \\
      \hline
    \end{tabular}
    \caption{Caso de uso 2.2.- Detener interpretación.}
\end{table}


%% Tabla Caso de uso 2.3 - Seleccionar nivel de depuración.
\begin{table}[!ht]
    \centering
    \begin{tabular}{|p{4cm}|p{11.5cm}|}
    \hline

    \textbf{Caso de uso} & \textit{Seleccionar nivel de depuración.}\\
    \hline

    \textbf{Objetivo} & \textit{Establecer el nivel de depuración que se llevará a cabo durante la siguiente interpretación.}\\
    \hline

    \textbf{Actores} & \textit{Usuario.}\\
    \hline

    \textbf{Disparador} & \textit{Este caso de uso se dispara cuando el usuario selecciona la opción correspondiente para seleccionar el nivel de
				  depuración.}\\
    \hline

    \textbf{Precondiciones} & \textit{Ninguna.}\\
    \hline

    \textbf{Descripción} & \textit{Establece el nivel de depuración para la siguiente interpretación.}\\
    \hline

    \multicolumn{2}{|c|}{\textbf{Curso normal de los eventos}}\\
    \hline

    \end{tabular}
    \begin{tabular}{|p{7.75cm}|p{7.75cm}|}
    \hspace{2cm}\textbf{Acción de los actores} & \hspace{1.75cm}\textbf{Respuesta del sistema}\\
    \hline

    \textit{2.- El usuario selecciona el nivel de depuración deseado.}
    &
    \textit{1.- El sistema da a elegir 3 niveles de depuración al usuario:}

    \textit{  - Sólo errores.}

    \textit{  - Información básica.}

    \textit{  - Información auxiliar.}

    \textit{3.- El sistema almacena el nuevo nivel de depuración.}
    \\
    \hline
    \end{tabular}

    \begin{tabular}{|p{15.9cm}|}
      \hspace{6cm}\textbf{Cursos alternativos}\\
      \hline     
	\textit{Ninguno.}
      \\
      \hline
    \end{tabular}
    \caption{Caso de uso 2.3.- Seleccionar nivel de depuración.}
\end{table}


\end{document}