\documentclass[a4paper,12pt]{article}
\usepackage[spanish]{babel} % Paquete para utilizar el español
\usepackage{palatino} % Para cambiar el tipo de letra

% Reconocimiento de caracteres del idioma Español en LaTeX.
\usepackage[utf8x]{inputenc} % Codificacion de entrada
\usepackage[T1]{fontenc} % Codificacion de fuente

% Cabecera de la "portada"
\title{\textbf{StickMotion:} Editor de posturas, posiciones y movimientos}
\author{\small\textit{Carmona Varo, Fernando; García García, José Manuel; López Fernandez, David;}\\
	\small\textit{Navas Torres, Francisco Javier; Porras Bueno, Javier}}

\begin{document}

  \maketitle % Crea el título y los autores

  \begin{abstract}
    En el presente documento se procederá a la descripción del trabajo a realizar en la parte de práctica de la asignatura Procesadores del
    Lenguaje, impartida por el profesor D. Enrique Yeguas Bolívar. Tal y como adelante el título de este informe, el trabajo consistirá en una
    aplicación que, a través de una serie de órdenes, ejecutará unos movimientos concretos y específicos.\\
  \end{abstract}

  \section{Introducción. Contexto del problema}
  La finalidad del software que se va a desarrollar es, tal y como se ha mencionado en el abstract de este documento, desarrollar una aplicación
  que ejecute una serie de órdenes que introduzca el usuario. Éste debe poseer ciertos conocimientos de programación para que le sea más fácil la
  utilización adecuada de sentencias, bucles, etc. \\
  
  La aplicación sólo reconoce y acepta sentencias desarrolladas para el \textbf{lenguaje de programación sticky}, el cual se abordará en posteriores 
  documentos de una manera más formal y técnica.\\
  
  \section{Arquitectura de la aplicación}
  La aplicación a desarrollar será posible ejecutarla sólo en local, por lo que no requerirá ningún tipo de conexión a Internet ni dependerá de otra
  aplicación, módulo o similar. De esta forma se hace sencillo su movilidad entre diversas máquinas y entre diversos sistemas operativos, pues al 
  estar desarrollada en Java permite una máxima compatibilidad, aumentando de esta forma el número potencial de usuarios que utilicen dicha 
  herramienta.\\
  
  \section{Descripción del stickman}
    \subsection{¿Cómo está formado?}
    El stickman pretende ser la representación sencilla del cuerpo de una persona, por lo que inicialmente posee sus mismas características y 
    articulaciones principales:
    \begin{itemize}
      \item Dos hombros que permiten la rotación de los brazos.
      \item Dos brazos que permiten la flexión/extensión por la articulación del codo.
      \item Dos piernas que permiten la flexión/extensión por la articulación de la rodilla.
      \item La cadera que permite la flexión/extensión por ella misma así como la rotación de las piernas.
      \item Un centro de gravedad o CDG, situado a la mitad aproximadamente del tronco.\\
    \end{itemize}
    
    \subsection{¿Qué es lo que puede hacer?}
    Nuestro stickman podrá realizar una serie de movimientos que podrán dotarlo de una funcionalidad suficiente para realizar la mayoría de acciones
    que se le ocurran al usuario:
    \begin{itemize}
      \item \textbf{\textit{Cabeza}}: podrá moverse horizontalmente hacia los lados, es decir, hacia un hombro y hacia otro.
      \item \textbf{\textit{Hombros}}: permiten movimientos de 360\textdegree para mover completamente los brazos en cualquier dirección y sentido.
      \item \textbf{\textit{Piernas}}: permiten movimientos de 360\textdegree para mover completamente las piernas en cualquier dirección y sentido.
      \item \textbf{\textit{Codos}}: permiten movimientos, como máximo, de 180\textdegree (lo que correspondería a la extensión completa del brazo).
      \item \textbf{\textit{Rodillas}}: permiten movimientos, como máximo, de 180\textdegree (lo que se corresponde con la extensión completa de la
	    pierna).\\
    \end{itemize}
    
    \subsection{¿Cómo lo puede hacer?}
    Para los movimientos descritos en el apartado anterior la aplicación cuenta con una serie de funciones que permitirán rotar cualquier miembro del
    stickman, así como posicionarlo en cualquier lugar de la ventana de animación.\\
    
    \subsection{¿Cómo es representado?}
    Como todo stickman, nuestro pequeño protagonista estará compuesta por, como se ha mencionado antes, una cabeza, un tronco, dos brazos y dos piernas,
    representados todos ellos por medio de líneas simples. Más adelante se estudiará la opción de mapearlo mediante texturas y dotarlo de un mayor
    realismo.\\
  
  \section{Descripción de la aplicación}
  La aplicación, de escritorio como se ha mencionado anteriormente, consiste en una interfaz gráfica que tendrá dos partes claramente diferenciadas:
  \begin{enumerate}
    \item La parte donde se dibujará el stickman y éste realizará todos sus movimientos.
    \item La parte donde se introducirán las sentencias o instrucciones; en definitiva, el uso de la aplicación. \\
  \end{enumerate}
  Para introducir las instrucciones se proveerán dos métodos:
  \begin{enumerate}
    \item Entrada estándar desde la propia aplicación, lo que facilita al usuario la prueba y testeo de acciones simples.
    \item Entrada desde fichero, lo que permite al usuario la prueba y testeo de un conjunto de instrucciones que conforman una acción más compleja.\\
  \end{enumerate}
  Además, la aplicación va a contar con las siguientes características:
  \begin{itemize}
    \item Para aquellos movimientos repetitivos o que posean cierto automatismo: bucles \textbf{while} y \textbf{for}.
    \item Para porciones de código condicionales: sentencias \textbf{if}.
    \item Creación y declaración de variables.
    \item Para el incremento y decremento de variables: operadores ``\textbf{++}'' y ``\textbf{--}''.
    \item Para la pausa entre movimientos: sentencias \textbf{wait}.
    \item Inclusión de librerías: fragmentos de código escrito en lenguaje sticky que implementan funciones concretas y específicas, aliviando así la
	  carga de trabajo al usuario, evitando de esta forma el tener que reinventar la rueda con código que ya ha sido desarrollado.
  \end{itemize}



  
  \section{Recursos}
  Los recursos con los que se cuenta para este trabajo son: 
  \begin{itemize}
    \item \textbf{Humano:} Los componentes del grupo, ya mencionados en el comienzo de este documento.
    \item \textbf{Software:} Se hará uso del entorno de desarrollo integrado Eclipse 3.4 junto con el plugin correspondiente de ANTLR.
    \item \textbf{Hardware:} Los equipos personales de los miembros que conforman el grupo.
  \end{itemize}


\end{document}
