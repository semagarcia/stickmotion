\chapter{Introducción}

   \section{Contexto del problema}
   La finalidad del software que se va a desarrollar es que sea capaz de ejecutar una serie de órdenes que introduzca el usuario. Estas ordenes
   permitiran mover y articular la representación de un humanoide. Éste debe poseer ciertos conocimientos de programación para que le sea más 
   fácil la utilización adecuada de sentencias, bucles, interfaz, etc.\\

   La aplicación sólo reconoce y acepta sentencias desarrolladas para el lenguaje de programación \textbf{\textit{Sticky}}, el cual se abordará
   en posteriores secciones de este documento.\\


   \section{Arquitectura de la aplicación}
   La aplicación a desarrollar será posible ejecutarla sólo en la máquina local, por lo que no requerirá ningún tipo de conexión a Internet ni
   dependerá de otra aplicación, módulo o similar. De esta forma se hace sencillo su movilidad entre diversas máquinas y entre diversos sistemas
   operativos, pues al estar desarrollada en Java permite una máxima compatibilidad, aumentando de esta forma el número potencial de usuarios que
   utilicen esta herramienta.\\


   \section{Descripción de StickMan}
   A continuación se detallan algunos aspectos que caracterizarán a nuestro StickMan.\\

      \subsection{¿Cómo está formado?}
      El stickman pretende ser la representación sencilla del cuerpo de una persona, por lo que inicialmente posee sus mismas características y 
      articulaciones principales: 
      \begin{itemize}
         \item Dos hombros que permiten la rotación de los brazos. 
         \item Dos brazos que permiten la flexión por la articulación del codo. 
         \item Dos piernas que permiten la flexión por la articulación de la rodilla. 
         \item La cadera que permite la rotación de las piernas. 
         \item Un \textit{centro de gravedad} o CDG, situado a la mitad aproximadamente del tronco.\\
      \end{itemize}


      \subsection{¿Qué puede hacer?}
      Stickman podrá realizar una serie de movimientos que pueden dotarlo de una funcionalidad suficiente para realizar la mayoría de acciones 
      que se le ocurran al usuario: 
      \begin{itemize}
         \item \textit{Cabeza}: puede rotar sobre sí misma.
         \item \textit{Hombros}: permiten movimientos de hasta 360º para mover completamente los brazos en cualquier dirección y sentido. 
         \item \textit{Piernas}: permiten movimientos de hasta 360º para mover completamente las piernas en cualquier dirección y sentido.
         \item \textit{Codos}: permiten movimientos de flexión, de 0 a 360º.
         \item \textit{Rodillas}: permiten movimientos de flexión de 0 a 360º.\\
      \end{itemize}


      \subsection{¿Cómo lo puede hacer?}
      Para los movimientos descritos en el apartado anterior la aplicación cuenta con una serie de funciones que permitirán rotar cualquier 
      miembro del stickman, así como posicionarlo en cualquier lugar de la ventana de animación.\\

   \section{Descripción de la aplicación}
   La aplicación, de escritorio como se ha mencionado anteriormente, consiste en una interfaz gráfica que tiene dos partes claramente diferenciadas: 
   \begin{enumerate}
      \item La parte donde se dibujará Stickman y éste realizará todos sus movimientos. 
      \item La parte donde se introducirán las sentencias o instrucciones; en definitiva, el uso de la aplicación.\\
   \end{enumerate}

   Para introducir las instrucciones se proveerán dos métodos: 
   \begin{enumerate}
      \item Entrada estándar desde la propia aplicación, lo que facilita al usuario la prueba y testeo de acciones simples. 
      \item Entrada desde fichero, lo que permite al usuario la prueba y testeo de un conjunto de instrucciones que conforman una acción más compleja.\\
   \end{enumerate}















