\chapter{Futuras Mejoras}
Aunque se han cumplido los objetivos propuestos al inicio del documento, a lo largo del proceso de desarrollo se han observado
algunas mejoras que proporcionarían una mayor funcionalidad en el ámbito del sistema.\\

A continuación se clasifican el conjunto de mejoras posibles que se podrían realizar en versiones posteriores de la aplicación:
\begin{enumerate}
   \item Inclusión de librerías, es decir, fragmentos de código escritos en lenguaje \textbf{\textit{Sticky}} que implementan funciones concretas 
         concretas  y específicas, aliviando así la carga de trabajo al usuario y evitando de esta forma el tener que reinventar la rueda
         con código que ya ha sido desarrollado anteriormente.
   \item Definición de funciones que fomenten la modularización y reutilización de los diferentes archivos de código.
   \item Desarrollar una herramienta que permita la elección de idioma de la interfaz.
   \item Proporcionar al usuario la opción de exportar resultados relacionados con el movimiento realizado por el objeto, tal y
         como la distancia recorrida, términos matemáticos avanzados, etc.
   \item Posibilidad de incrustar el sistema en un servidor web para que sea accedido mediante cualquier navegador web, sin 
         necesidad de poseer el programa localmente, sino que sea ofrecido como un servicio dentro de la nube.
   \item Al hacer doble click en la depuración mostrada al ejecutar código, ir a la línea en el código correspondiente.
\end{enumerate}
