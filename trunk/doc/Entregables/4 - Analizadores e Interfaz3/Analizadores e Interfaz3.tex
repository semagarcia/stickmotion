\documentclass[a4paper,12pt]{article}
\usepackage[spanish]{babel} % Paquete para utilizar el español
\usepackage{palatino} % Para cambiar el tipo de letra

% Reconocimiento de caracteres del idioma Español en LaTeX.
\usepackage[utf8x]{inputenc} % Codificacion de entrada
\usepackage[T1]{fontenc} % Codificacion de fuente
\usepackage{graphicx}

% Cabecera de la "portada"
\title{\textbf{StickMotion:} Editor de posturas, posiciones y movimientos}
\author{\small\textit{Carmona Varo, Fernando; García García, José Manuel; López Fernandez, David;}\\
	\small\textit{Navas Torres, Francisco Javier; Porras Bueno, Javier}}

\begin{document}

  \maketitle % Crea el título y los autores

  \begin{abstract}
    El presente documento se corresponde con el entregable nº 4. El cliente recibirá ... \\
  \end{abstract}

  \section{Analizador Léxico}
  A continuación se describe la gramática de Sticky que se ha desarrollado:
  
  
  \section{Analizador Sintáctico}


  \section{Analizador Semántico}
  

  \section{Nuevas funcionalidades de la Interfaz}
  % Hablar de que si se pulsa CTRL + F se abre para buscar palabras. también cuando te pones en una variable, se resaltan todas,
  % se inserta solo el paréntesis al escribir uno izquierdo, se resaltan los parentesis/llaves emparejados....
\end{document}
